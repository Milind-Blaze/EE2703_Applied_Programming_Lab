
% Default to the notebook output style

    


% Inherit from the specified cell style.




    
\documentclass[a4paper, 12pt, margin= 1.25cm ]{article}

    \usepackage{multicol}
    \usepackage{amsmath}
    \usepackage[T1]{fontenc}
    % Nicer default font (+ math font) than Computer Modern for most use cases
    \usepackage{mathpazo}

    % Basic figure setup, for now with no caption control since it's done
    % automatically by Pandoc (which extracts ![](path) syntax from Markdown).
    \usepackage{graphicx}
    % We will generate all images so they have a width \maxwidth. This means
    % that they will get their normal width if they fit onto the page, but
    % are scaled down if they would overflow the margins.
    \makeatletter
    \def\maxwidth{\ifdim\Gin@nat@width>\linewidth\linewidth
    \else\Gin@nat@width\fi}
    \makeatother
    \let\Oldincludegraphics\includegraphics
    % Set max figure width to be 80% of text width, for now hardcoded.
    \renewcommand{\includegraphics}[1]{\Oldincludegraphics[width=.8\maxwidth]{#1}}
    % Ensure that by default, figures have no caption (until we provide a
    % proper Figure object with a Caption API and a way to capture that
    % in the conversion process - todo).
    \usepackage{caption}
    \DeclareCaptionLabelFormat{nolabel}{}
    \captionsetup{labelformat=nolabel}

    \usepackage{adjustbox} % Used to constrain images to a maximum size 
    \usepackage{xcolor} % Allow colors to be defined
    \usepackage{enumerate} % Needed for markdown enumerations to work
    \usepackage{geometry} % Used to adjust the document margins
    \usepackage{amsmath} % Equations
    \usepackage{amssymb} % Equations
    \usepackage{textcomp} % defines textquotesingle
    % Hack from http://tex.stackexchange.com/a/47451/13684:
    \AtBeginDocument{%
        \def\PYZsq{\textquotesingle}% Upright quotes in Pygmentized code
    }
    \usepackage{upquote} % Upright quotes for verbatim code
    \usepackage{eurosym} % defines \euro
    \usepackage[mathletters]{ucs} % Extended unicode (utf-8) support
    \usepackage[utf8x]{inputenc} % Allow utf-8 characters in the tex document
    \usepackage{fancyvrb} % verbatim replacement that allows latex
    \usepackage{grffile} % extends the file name processing of package graphics 
                         % to support a larger range 
    % The hyperref package gives us a pdf with properly built
    % internal navigation ('pdf bookmarks' for the table of contents,
    % internal cross-reference links, web links for URLs, etc.)
    \usepackage{hyperref}
    \usepackage{longtable} % longtable support required by pandoc >1.10
    \usepackage{booktabs}  % table support for pandoc > 1.12.2
    \usepackage[inline]{enumitem} % IRkernel/repr support (it uses the enumerate* environment)
    \usepackage[normalem]{ulem} % ulem is needed to support strikethroughs (\sout)
                                % normalem makes italics be italics, not underlines
    

    
    
    % Colors for the hyperref package
    \definecolor{urlcolor}{rgb}{0,.145,.698}
    \definecolor{linkcolor}{rgb}{.71,0.21,0.01}
    \definecolor{citecolor}{rgb}{.12,.54,.11}

    % ANSI colors
    \definecolor{ansi-black}{HTML}{3E424D}
    \definecolor{ansi-black-intense}{HTML}{282C36}
    \definecolor{ansi-red}{HTML}{E75C58}
    \definecolor{ansi-red-intense}{HTML}{B22B31}
    \definecolor{ansi-green}{HTML}{00A250}
    \definecolor{ansi-green-intense}{HTML}{007427}
    \definecolor{ansi-yellow}{HTML}{DDB62B}
    \definecolor{ansi-yellow-intense}{HTML}{B27D12}
    \definecolor{ansi-blue}{HTML}{208FFB}
    \definecolor{ansi-blue-intense}{HTML}{0065CA}
    \definecolor{ansi-magenta}{HTML}{D160C4}
    \definecolor{ansi-magenta-intense}{HTML}{A03196}
    \definecolor{ansi-cyan}{HTML}{60C6C8}
    \definecolor{ansi-cyan-intense}{HTML}{258F8F}
    \definecolor{ansi-white}{HTML}{C5C1B4}
    \definecolor{ansi-white-intense}{HTML}{A1A6B2}

    % commands and environments needed by pandoc snippets
    % extracted from the output of `pandoc -s`
    \providecommand{\tightlist}{%
      \setlength{\itemsep}{0pt}\setlength{\parskip}{0pt}}
    \DefineVerbatimEnvironment{Highlighting}{Verbatim}{commandchars=\\\{\}}
    % Add ',fontsize=\small' for more characters per line
    \newenvironment{Shaded}{}{}
    \newcommand{\KeywordTok}[1]{\textcolor[rgb]{0.00,0.44,0.13}{\textbf{{#1}}}}
    \newcommand{\DataTypeTok}[1]{\textcolor[rgb]{0.56,0.13,0.00}{{#1}}}
    \newcommand{\DecValTok}[1]{\textcolor[rgb]{0.25,0.63,0.44}{{#1}}}
    \newcommand{\BaseNTok}[1]{\textcolor[rgb]{0.25,0.63,0.44}{{#1}}}
    \newcommand{\FloatTok}[1]{\textcolor[rgb]{0.25,0.63,0.44}{{#1}}}
    \newcommand{\CharTok}[1]{\textcolor[rgb]{0.25,0.44,0.63}{{#1}}}
    \newcommand{\StringTok}[1]{\textcolor[rgb]{0.25,0.44,0.63}{{#1}}}
    \newcommand{\CommentTok}[1]{\textcolor[rgb]{0.38,0.63,0.69}{\textit{{#1}}}}
    \newcommand{\OtherTok}[1]{\textcolor[rgb]{0.00,0.44,0.13}{{#1}}}
    \newcommand{\AlertTok}[1]{\textcolor[rgb]{1.00,0.00,0.00}{\textbf{{#1}}}}
    \newcommand{\FunctionTok}[1]{\textcolor[rgb]{0.02,0.16,0.49}{{#1}}}
    \newcommand{\RegionMarkerTok}[1]{{#1}}
    \newcommand{\ErrorTok}[1]{\textcolor[rgb]{1.00,0.00,0.00}{\textbf{{#1}}}}
    \newcommand{\NormalTok}[1]{{#1}}
    
    % Additional commands for more recent versions of Pandoc
    \newcommand{\ConstantTok}[1]{\textcolor[rgb]{0.53,0.00,0.00}{{#1}}}
    \newcommand{\SpecialCharTok}[1]{\textcolor[rgb]{0.25,0.44,0.63}{{#1}}}
    \newcommand{\VerbatimStringTok}[1]{\textcolor[rgb]{0.25,0.44,0.63}{{#1}}}
    \newcommand{\SpecialStringTok}[1]{\textcolor[rgb]{0.73,0.40,0.53}{{#1}}}
    \newcommand{\ImportTok}[1]{{#1}}
    \newcommand{\DocumentationTok}[1]{\textcolor[rgb]{0.73,0.13,0.13}{\textit{{#1}}}}
    \newcommand{\AnnotationTok}[1]{\textcolor[rgb]{0.38,0.63,0.69}{\textbf{\textit{{#1}}}}}
    \newcommand{\CommentVarTok}[1]{\textcolor[rgb]{0.38,0.63,0.69}{\textbf{\textit{{#1}}}}}
    \newcommand{\VariableTok}[1]{\textcolor[rgb]{0.10,0.09,0.49}{{#1}}}
    \newcommand{\ControlFlowTok}[1]{\textcolor[rgb]{0.00,0.44,0.13}{\textbf{{#1}}}}
    \newcommand{\OperatorTok}[1]{\textcolor[rgb]{0.40,0.40,0.40}{{#1}}}
    \newcommand{\BuiltInTok}[1]{{#1}}
    \newcommand{\ExtensionTok}[1]{{#1}}
    \newcommand{\PreprocessorTok}[1]{\textcolor[rgb]{0.74,0.48,0.00}{{#1}}}
    \newcommand{\AttributeTok}[1]{\textcolor[rgb]{0.49,0.56,0.16}{{#1}}}
    \newcommand{\InformationTok}[1]{\textcolor[rgb]{0.38,0.63,0.69}{\textbf{\textit{{#1}}}}}
    \newcommand{\WarningTok}[1]{\textcolor[rgb]{0.38,0.63,0.69}{\textbf{\textit{{#1}}}}}
    
    
    % Define a nice break command that doesn't care if a line doesn't already
    % exist.
    \def\br{\hspace*{\fill} \\* }
    % Math Jax compatability definitions
    \def\gt{>}
    \def\lt{<}
    % Document parameters
    \title{Fourier Series}
    \date{17-02-2018}
    \author{Milind Kumar V\\ EE16B025}
    
    
    

    % Pygments definitions
    
\makeatletter
\def\PY@reset{\let\PY@it=\relax \let\PY@bf=\relax%
    \let\PY@ul=\relax \let\PY@tc=\relax%
    \let\PY@bc=\relax \let\PY@ff=\relax}
\def\PY@tok#1{\csname PY@tok@#1\endcsname}
\def\PY@toks#1+{\ifx\relax#1\empty\else%
    \PY@tok{#1}\expandafter\PY@toks\fi}
\def\PY@do#1{\PY@bc{\PY@tc{\PY@ul{%
    \PY@it{\PY@bf{\PY@ff{#1}}}}}}}
\def\PY#1#2{\PY@reset\PY@toks#1+\relax+\PY@do{#2}}

\expandafter\def\csname PY@tok@gd\endcsname{\def\PY@tc##1{\textcolor[rgb]{0.63,0.00,0.00}{##1}}}
\expandafter\def\csname PY@tok@gu\endcsname{\let\PY@bf=\textbf\def\PY@tc##1{\textcolor[rgb]{0.50,0.00,0.50}{##1}}}
\expandafter\def\csname PY@tok@gt\endcsname{\def\PY@tc##1{\textcolor[rgb]{0.00,0.27,0.87}{##1}}}
\expandafter\def\csname PY@tok@gs\endcsname{\let\PY@bf=\textbf}
\expandafter\def\csname PY@tok@gr\endcsname{\def\PY@tc##1{\textcolor[rgb]{1.00,0.00,0.00}{##1}}}
\expandafter\def\csname PY@tok@cm\endcsname{\let\PY@it=\textit\def\PY@tc##1{\textcolor[rgb]{0.25,0.50,0.50}{##1}}}
\expandafter\def\csname PY@tok@vg\endcsname{\def\PY@tc##1{\textcolor[rgb]{0.10,0.09,0.49}{##1}}}
\expandafter\def\csname PY@tok@vi\endcsname{\def\PY@tc##1{\textcolor[rgb]{0.10,0.09,0.49}{##1}}}
\expandafter\def\csname PY@tok@vm\endcsname{\def\PY@tc##1{\textcolor[rgb]{0.10,0.09,0.49}{##1}}}
\expandafter\def\csname PY@tok@mh\endcsname{\def\PY@tc##1{\textcolor[rgb]{0.40,0.40,0.40}{##1}}}
\expandafter\def\csname PY@tok@cs\endcsname{\let\PY@it=\textit\def\PY@tc##1{\textcolor[rgb]{0.25,0.50,0.50}{##1}}}
\expandafter\def\csname PY@tok@ge\endcsname{\let\PY@it=\textit}
\expandafter\def\csname PY@tok@vc\endcsname{\def\PY@tc##1{\textcolor[rgb]{0.10,0.09,0.49}{##1}}}
\expandafter\def\csname PY@tok@il\endcsname{\def\PY@tc##1{\textcolor[rgb]{0.40,0.40,0.40}{##1}}}
\expandafter\def\csname PY@tok@go\endcsname{\def\PY@tc##1{\textcolor[rgb]{0.53,0.53,0.53}{##1}}}
\expandafter\def\csname PY@tok@cp\endcsname{\def\PY@tc##1{\textcolor[rgb]{0.74,0.48,0.00}{##1}}}
\expandafter\def\csname PY@tok@gi\endcsname{\def\PY@tc##1{\textcolor[rgb]{0.00,0.63,0.00}{##1}}}
\expandafter\def\csname PY@tok@gh\endcsname{\let\PY@bf=\textbf\def\PY@tc##1{\textcolor[rgb]{0.00,0.00,0.50}{##1}}}
\expandafter\def\csname PY@tok@ni\endcsname{\let\PY@bf=\textbf\def\PY@tc##1{\textcolor[rgb]{0.60,0.60,0.60}{##1}}}
\expandafter\def\csname PY@tok@nl\endcsname{\def\PY@tc##1{\textcolor[rgb]{0.63,0.63,0.00}{##1}}}
\expandafter\def\csname PY@tok@nn\endcsname{\let\PY@bf=\textbf\def\PY@tc##1{\textcolor[rgb]{0.00,0.00,1.00}{##1}}}
\expandafter\def\csname PY@tok@no\endcsname{\def\PY@tc##1{\textcolor[rgb]{0.53,0.00,0.00}{##1}}}
\expandafter\def\csname PY@tok@na\endcsname{\def\PY@tc##1{\textcolor[rgb]{0.49,0.56,0.16}{##1}}}
\expandafter\def\csname PY@tok@nb\endcsname{\def\PY@tc##1{\textcolor[rgb]{0.00,0.50,0.00}{##1}}}
\expandafter\def\csname PY@tok@nc\endcsname{\let\PY@bf=\textbf\def\PY@tc##1{\textcolor[rgb]{0.00,0.00,1.00}{##1}}}
\expandafter\def\csname PY@tok@nd\endcsname{\def\PY@tc##1{\textcolor[rgb]{0.67,0.13,1.00}{##1}}}
\expandafter\def\csname PY@tok@ne\endcsname{\let\PY@bf=\textbf\def\PY@tc##1{\textcolor[rgb]{0.82,0.25,0.23}{##1}}}
\expandafter\def\csname PY@tok@nf\endcsname{\def\PY@tc##1{\textcolor[rgb]{0.00,0.00,1.00}{##1}}}
\expandafter\def\csname PY@tok@si\endcsname{\let\PY@bf=\textbf\def\PY@tc##1{\textcolor[rgb]{0.73,0.40,0.53}{##1}}}
\expandafter\def\csname PY@tok@s2\endcsname{\def\PY@tc##1{\textcolor[rgb]{0.73,0.13,0.13}{##1}}}
\expandafter\def\csname PY@tok@nt\endcsname{\let\PY@bf=\textbf\def\PY@tc##1{\textcolor[rgb]{0.00,0.50,0.00}{##1}}}
\expandafter\def\csname PY@tok@nv\endcsname{\def\PY@tc##1{\textcolor[rgb]{0.10,0.09,0.49}{##1}}}
\expandafter\def\csname PY@tok@s1\endcsname{\def\PY@tc##1{\textcolor[rgb]{0.73,0.13,0.13}{##1}}}
\expandafter\def\csname PY@tok@dl\endcsname{\def\PY@tc##1{\textcolor[rgb]{0.73,0.13,0.13}{##1}}}
\expandafter\def\csname PY@tok@ch\endcsname{\let\PY@it=\textit\def\PY@tc##1{\textcolor[rgb]{0.25,0.50,0.50}{##1}}}
\expandafter\def\csname PY@tok@m\endcsname{\def\PY@tc##1{\textcolor[rgb]{0.40,0.40,0.40}{##1}}}
\expandafter\def\csname PY@tok@gp\endcsname{\let\PY@bf=\textbf\def\PY@tc##1{\textcolor[rgb]{0.00,0.00,0.50}{##1}}}
\expandafter\def\csname PY@tok@sh\endcsname{\def\PY@tc##1{\textcolor[rgb]{0.73,0.13,0.13}{##1}}}
\expandafter\def\csname PY@tok@ow\endcsname{\let\PY@bf=\textbf\def\PY@tc##1{\textcolor[rgb]{0.67,0.13,1.00}{##1}}}
\expandafter\def\csname PY@tok@sx\endcsname{\def\PY@tc##1{\textcolor[rgb]{0.00,0.50,0.00}{##1}}}
\expandafter\def\csname PY@tok@bp\endcsname{\def\PY@tc##1{\textcolor[rgb]{0.00,0.50,0.00}{##1}}}
\expandafter\def\csname PY@tok@c1\endcsname{\let\PY@it=\textit\def\PY@tc##1{\textcolor[rgb]{0.25,0.50,0.50}{##1}}}
\expandafter\def\csname PY@tok@fm\endcsname{\def\PY@tc##1{\textcolor[rgb]{0.00,0.00,1.00}{##1}}}
\expandafter\def\csname PY@tok@o\endcsname{\def\PY@tc##1{\textcolor[rgb]{0.40,0.40,0.40}{##1}}}
\expandafter\def\csname PY@tok@kc\endcsname{\let\PY@bf=\textbf\def\PY@tc##1{\textcolor[rgb]{0.00,0.50,0.00}{##1}}}
\expandafter\def\csname PY@tok@c\endcsname{\let\PY@it=\textit\def\PY@tc##1{\textcolor[rgb]{0.25,0.50,0.50}{##1}}}
\expandafter\def\csname PY@tok@mf\endcsname{\def\PY@tc##1{\textcolor[rgb]{0.40,0.40,0.40}{##1}}}
\expandafter\def\csname PY@tok@err\endcsname{\def\PY@bc##1{\setlength{\fboxsep}{0pt}\fcolorbox[rgb]{1.00,0.00,0.00}{1,1,1}{\strut ##1}}}
\expandafter\def\csname PY@tok@mb\endcsname{\def\PY@tc##1{\textcolor[rgb]{0.40,0.40,0.40}{##1}}}
\expandafter\def\csname PY@tok@ss\endcsname{\def\PY@tc##1{\textcolor[rgb]{0.10,0.09,0.49}{##1}}}
\expandafter\def\csname PY@tok@sr\endcsname{\def\PY@tc##1{\textcolor[rgb]{0.73,0.40,0.53}{##1}}}
\expandafter\def\csname PY@tok@mo\endcsname{\def\PY@tc##1{\textcolor[rgb]{0.40,0.40,0.40}{##1}}}
\expandafter\def\csname PY@tok@kd\endcsname{\let\PY@bf=\textbf\def\PY@tc##1{\textcolor[rgb]{0.00,0.50,0.00}{##1}}}
\expandafter\def\csname PY@tok@mi\endcsname{\def\PY@tc##1{\textcolor[rgb]{0.40,0.40,0.40}{##1}}}
\expandafter\def\csname PY@tok@kn\endcsname{\let\PY@bf=\textbf\def\PY@tc##1{\textcolor[rgb]{0.00,0.50,0.00}{##1}}}
\expandafter\def\csname PY@tok@cpf\endcsname{\let\PY@it=\textit\def\PY@tc##1{\textcolor[rgb]{0.25,0.50,0.50}{##1}}}
\expandafter\def\csname PY@tok@kr\endcsname{\let\PY@bf=\textbf\def\PY@tc##1{\textcolor[rgb]{0.00,0.50,0.00}{##1}}}
\expandafter\def\csname PY@tok@s\endcsname{\def\PY@tc##1{\textcolor[rgb]{0.73,0.13,0.13}{##1}}}
\expandafter\def\csname PY@tok@kp\endcsname{\def\PY@tc##1{\textcolor[rgb]{0.00,0.50,0.00}{##1}}}
\expandafter\def\csname PY@tok@w\endcsname{\def\PY@tc##1{\textcolor[rgb]{0.73,0.73,0.73}{##1}}}
\expandafter\def\csname PY@tok@kt\endcsname{\def\PY@tc##1{\textcolor[rgb]{0.69,0.00,0.25}{##1}}}
\expandafter\def\csname PY@tok@sc\endcsname{\def\PY@tc##1{\textcolor[rgb]{0.73,0.13,0.13}{##1}}}
\expandafter\def\csname PY@tok@sb\endcsname{\def\PY@tc##1{\textcolor[rgb]{0.73,0.13,0.13}{##1}}}
\expandafter\def\csname PY@tok@sa\endcsname{\def\PY@tc##1{\textcolor[rgb]{0.73,0.13,0.13}{##1}}}
\expandafter\def\csname PY@tok@k\endcsname{\let\PY@bf=\textbf\def\PY@tc##1{\textcolor[rgb]{0.00,0.50,0.00}{##1}}}
\expandafter\def\csname PY@tok@se\endcsname{\let\PY@bf=\textbf\def\PY@tc##1{\textcolor[rgb]{0.73,0.40,0.13}{##1}}}
\expandafter\def\csname PY@tok@sd\endcsname{\let\PY@it=\textit\def\PY@tc##1{\textcolor[rgb]{0.73,0.13,0.13}{##1}}}

\def\PYZbs{\char`\\}
\def\PYZus{\char`\_}
\def\PYZob{\char`\{}
\def\PYZcb{\char`\}}
\def\PYZca{\char`\^}
\def\PYZam{\char`\&}
\def\PYZlt{\char`\<}
\def\PYZgt{\char`\>}
\def\PYZsh{\char`\#}
\def\PYZpc{\char`\%}
\def\PYZdl{\char`\$}
\def\PYZhy{\char`\-}
\def\PYZsq{\char`\'}
\def\PYZdq{\char`\"}
\def\PYZti{\char`\~}
% for compatibility with earlier versions
\def\PYZat{@}
\def\PYZlb{[}
\def\PYZrb{]}
\makeatother


    % Exact colors from NB
    \definecolor{incolor}{rgb}{0.0, 0.0, 0.5}
    \definecolor{outcolor}{rgb}{0.545, 0.0, 0.0}



    
    % Prevent overflowing lines due to hard-to-break entities
    \sloppy 
    % Setup hyperref package
    \hypersetup{
      breaklinks=true,  % so long urls are correctly broken across lines
      colorlinks=true,
      urlcolor=urlcolor,
      linkcolor=linkcolor,
      citecolor=citecolor,
      }
    % Slightly bigger margins than the latex defaults
    
    \geometry{verbose,tmargin=1in,bmargin=1in,lmargin=1in,rmargin=1in}
    
    

    \begin{document}
    
    
    \maketitle
    
 \begin{multicols}{2}   

    \begin{abstract}
    

This report explores the usage of scientific python to determine the
Fourier coefficients of selected functions. Further, it also looks at
the convergence of fourier series and attempts to draw conclusions about
the convergence of the fourier series of piecewise smooth functions and
the Gibbs phenomenon. Two approaches are used to determine the fourier
coefficients- integration using the quad function and the least squares
approach.
    \end{abstract}
    \section{Introduction}\label{introduction}

Two functions are considered for the following fourier analysis- the
exponential function $e^x$ and the $cos(cos(x))$ function. The former
function is extended periodically to the real number line and is $2\pi$
periodic. The $cos(cos(x))$ function is periodic and continuous with the
period $2\pi$. The $quad$ function from the Scipy library is used to
obtain the first 51 fourier coefficients of the functions. These are
plotted and analysed. This is followed by the use of the least squares
approach to do the same. Finally the reconstructed functions are
analysed.

    \section{Methods and results}\label{methods-and-results}

We begin by making the necessary imports- the numpy and matplotlib
libraries and $quad$ from the scipy library. Further we set all image
sizes to 10 x 8. The functions necessary to obtain the fourier
coefficients are defined the next piece of code. These include the
$e^x$, $cos(cos(x))$, $e^{x}cos(kx)$, $e^{x}six(kx)$,
$cos(cos(x))cos(kx)$, $cos(cos(x))sin(kx)$. The latter eight functions
are integrated to obtaing the $kth$ fourier coefficients $a_k$ and
$b_k$.

\end{multicols}




    \begin{Verbatim}[commandchars=\\\{\}]
{\color{incolor}In [{\color{incolor}139}]:} \PY{k+kn}{from} \PY{n+nn}{\PYZus{}\PYZus{}future\PYZus{}\PYZus{}} \PY{k+kn}{import} \PY{n}{division}
          \PY{o}{\PYZpc{}} \PY{n}{matplotlib} \PY{n}{inline}
          
          \PY{k+kn}{import} \PY{n+nn}{numpy} \PY{k+kn}{as} \PY{n+nn}{np}
          \PY{k+kn}{import} \PY{n+nn}{matplotlib}
          \PY{k+kn}{from} \PY{n+nn}{matplotlib} \PY{k+kn}{import} \PY{n}{pyplot} \PY{k}{as} \PY{n}{plt}
          \PY{k+kn}{from} \PY{n+nn}{scipy.integrate} \PY{k+kn}{import} \PY{n}{quad}
          
          \PY{n}{size}\PY{o}{=}\PY{p}{(}\PY{l+m+mi}{10}\PY{p}{,}\PY{l+m+mi}{8}\PY{p}{)}
\end{Verbatim}


    \begin{Verbatim}[commandchars=\\\{\}]
{\color{incolor}In [{\color{incolor}140}]:} \PY{c+c1}{\PYZsh{}functions}
          
          \PY{k}{def} \PY{n+nf}{exponential}\PY{p}{(}\PY{n}{x}\PY{p}{)}\PY{p}{:}
              \PY{k}{return} \PY{n}{np}\PY{o}{.}\PY{n}{exp}\PY{p}{(}\PY{n}{x}\PY{p}{)}
          
          \PY{k}{def} \PY{n+nf}{coscos}\PY{p}{(}\PY{n}{x}\PY{p}{)}\PY{p}{:}
              \PY{k}{return} \PY{n}{np}\PY{o}{.}\PY{n}{cos}\PY{p}{(}\PY{n}{np}\PY{o}{.}\PY{n}{cos}\PY{p}{(}\PY{n}{x}\PY{p}{)}\PY{p}{)}
          
          \PY{k}{def} \PY{n+nf}{expcos}\PY{p}{(}\PY{n}{x}\PY{p}{,}\PY{n}{k}\PY{p}{)}\PY{p}{:}
              \PY{k}{return} \PY{p}{(}\PY{n}{exponential}\PY{p}{(}\PY{n}{x}\PY{p}{)}\PY{p}{)}\PY{o}{*}\PY{n}{np}\PY{o}{.}\PY{n}{cos}\PY{p}{(}\PY{n}{k}\PY{o}{*}\PY{n}{x}\PY{p}{)}
          
          \PY{k}{def} \PY{n+nf}{expsin}\PY{p}{(}\PY{n}{x}\PY{p}{,}\PY{n}{k}\PY{p}{)}\PY{p}{:}
              \PY{k}{return} \PY{p}{(}\PY{n}{exponential}\PY{p}{(}\PY{n}{x}\PY{p}{)}\PY{p}{)}\PY{o}{*}\PY{n}{np}\PY{o}{.}\PY{n}{sin}\PY{p}{(}\PY{n}{k}\PY{o}{*}\PY{n}{x}\PY{p}{)}
          
          \PY{k}{def} \PY{n+nf}{cccos}\PY{p}{(}\PY{n}{x}\PY{p}{,}\PY{n}{k}\PY{p}{)}\PY{p}{:}
              \PY{k}{return} \PY{p}{(}\PY{n}{coscos}\PY{p}{(}\PY{n}{x}\PY{p}{)}\PY{p}{)}\PY{o}{*}\PY{n}{np}\PY{o}{.}\PY{n}{cos}\PY{p}{(}\PY{n}{k}\PY{o}{*}\PY{n}{x}\PY{p}{)}
          
          \PY{k}{def} \PY{n+nf}{ccsin}\PY{p}{(}\PY{n}{x}\PY{p}{,}\PY{n}{k}\PY{p}{)}\PY{p}{:}
              \PY{k}{return} \PY{p}{(}\PY{n}{coscos}\PY{p}{(}\PY{n}{x}\PY{p}{)}\PY{p}{)}\PY{o}{*}\PY{n}{np}\PY{o}{.}\PY{n}{sin}\PY{p}{(}\PY{n}{k}\PY{o}{*}\PY{n}{x}\PY{p}{)}
\end{Verbatim}

\begin{multicols}{2}


    The following two functions are the focus of this work and are
inherently $2\pi$ periodic or extended to be so on the real number line.

\begin{align}
f_1(x) &= e^x\\
f_2(x ) &= cos(cos(x))
\end{align}

The exponential function is plotted on the semilog scale and the
$cos(cos(x))$ function on the linear scale over the interval
$[-2\pi, 4\pi)$. The input vector to the predefined functions is created
using the $arange$ function in numpy.

\end{multicols}

    \begin{Verbatim}[commandchars=\\\{\}]
{\color{incolor}In [{\color{incolor}141}]:} \PY{n}{valuerange}\PY{o}{=} \PY{n}{np}\PY{o}{.}\PY{n}{arange}\PY{p}{(}\PY{o}{\PYZhy{}}\PY{l+m+mi}{2}\PY{o}{*}\PY{n}{np}\PY{o}{.}\PY{n}{pi}\PY{p}{,} \PY{l+m+mi}{4}\PY{o}{*}\PY{n}{np}\PY{o}{.}\PY{n}{pi}\PY{p}{,} \PY{l+m+mf}{0.1}\PY{p}{)}
          \PY{n}{periodicfunc}\PY{o}{=}\PY{p}{[}\PY{p}{]}
          \PY{k}{for} \PY{n}{i} \PY{o+ow}{in} \PY{n+nb}{range}\PY{p}{(}\PY{l+m+mi}{0}\PY{p}{,}\PY{n+nb}{len}\PY{p}{(}\PY{n}{valuerange}\PY{p}{)}\PY{p}{)}\PY{p}{:}
              \PY{k}{if} \PY{n}{valuerange}\PY{p}{[}\PY{n}{i}\PY{p}{]}\PY{o}{\PYZlt{}}\PY{o}{=}\PY{l+m+mi}{0}\PY{p}{:}
                  \PY{n}{periodicfunc}\PY{o}{.}\PY{n}{append}\PY{p}{(}\PY{n}{exponential}\PY{p}{(}\PY{n}{valuerange}\PY{p}{[}\PY{n}{i}\PY{p}{]}\PY{o}{+}\PY{l+m+mi}{2}\PY{o}{*}\PY{n}{np}\PY{o}{.}\PY{n}{pi}\PY{p}{)}\PY{p}{)}
              \PY{k}{if} \PY{l+m+mi}{0}\PY{o}{\PYZlt{}}\PY{n}{valuerange}\PY{p}{[}\PY{n}{i}\PY{p}{]}\PY{o}{\PYZlt{}}\PY{o}{=}\PY{l+m+mi}{2}\PY{o}{*}\PY{n}{np}\PY{o}{.}\PY{n}{pi}\PY{p}{:}
                  \PY{n}{periodicfunc}\PY{o}{.}\PY{n}{append}\PY{p}{(}\PY{n}{exponential}\PY{p}{(}\PY{n}{valuerange}\PY{p}{[}\PY{n}{i}\PY{p}{]}\PY{p}{)}\PY{p}{)}        
              \PY{k}{if} \PY{l+m+mi}{2}\PY{o}{*}\PY{n}{np}\PY{o}{.}\PY{n}{pi}\PY{o}{\PYZlt{}}\PY{n}{valuerange}\PY{p}{[}\PY{n}{i}\PY{p}{]}\PY{o}{\PYZlt{}}\PY{l+m+mi}{4}\PY{o}{*}\PY{n}{np}\PY{o}{.}\PY{n}{pi}\PY{p}{:}
                  \PY{n}{periodicfunc}\PY{o}{.}\PY{n}{append}\PY{p}{(}\PY{n}{exponential}\PY{p}{(}\PY{n}{valuerange}\PY{p}{[}\PY{n}{i}\PY{p}{]}\PY{o}{\PYZhy{}}\PY{l+m+mi}{2}\PY{o}{*}\PY{n}{np}\PY{o}{.}\PY{n}{pi}\PY{p}{)}\PY{p}{)}
                  
          \PY{n}{fig1}\PY{o}{=} \PY{n}{plt}\PY{o}{.}\PY{n}{figure}\PY{p}{(}\PY{l+m+mi}{1}\PY{p}{,} \PY{n}{figsize}\PY{o}{=} \PY{n}{size}\PY{p}{)}
          \PY{n}{axes1}\PY{o}{=} \PY{n}{fig1}\PY{o}{.}\PY{n}{add\PYZus{}subplot}\PY{p}{(}\PY{l+m+mi}{1}\PY{p}{,}\PY{l+m+mi}{1}\PY{p}{,}\PY{l+m+mi}{1}\PY{p}{)}
          \PY{n}{axes1}\PY{o}{.}\PY{n}{grid}\PY{p}{(}\PY{n+nb+bp}{True}\PY{p}{)}
          \PY{n}{axes1}\PY{o}{.}\PY{n}{set\PYZus{}xlabel}\PY{p}{(}\PY{l+s+s2}{\PYZdq{}}\PY{l+s+s2}{\PYZdl{}x\PYZdl{}}\PY{l+s+s2}{\PYZdq{}}\PY{p}{)}
          \PY{n}{axes1}\PY{o}{.}\PY{n}{set\PYZus{}ylabel}\PY{p}{(}\PY{l+s+s2}{\PYZdq{}}\PY{l+s+s2}{\PYZdl{}e\PYZca{}x\PYZdl{}}\PY{l+s+s2}{\PYZdq{}}\PY{p}{)}
          \PY{n}{axes1}\PY{o}{.}\PY{n}{set\PYZus{}title}\PY{p}{(}\PY{l+s+s2}{\PYZdq{}}\PY{l+s+s2}{\PYZdl{}e\PYZca{}x\PYZdl{}}\PY{l+s+s2}{\PYZdq{}}\PY{p}{)}
          \PY{n}{random}\PY{o}{=}\PY{n}{axes1}\PY{o}{.}\PY{n}{semilogy}\PY{p}{(}\PY{n}{valuerange}\PY{p}{,} \PY{n}{exponential}\PY{p}{(}\PY{n}{valuerange}\PY{p}{)}\PY{p}{,}\PY{l+s+s2}{\PYZdq{}}\PY{l+s+s2}{ro}\PY{l+s+s2}{\PYZdq{}}\PY{p}{)}
          \PY{n}{random}\PY{o}{=}\PY{n}{axes1}\PY{o}{.}\PY{n}{semilogy}\PY{p}{(}\PY{n}{valuerange}\PY{p}{,} \PY{n}{periodicfunc}\PY{p}{,}\PY{l+s+s2}{\PYZdq{}}\PY{l+s+s2}{bo}\PY{l+s+s2}{\PYZdq{}}\PY{p}{)}
\end{Verbatim}


    \begin{center}
    \adjustimage{max size={0.9\linewidth}{0.9\paperheight}}{output_6_0.png}
    \end{center}
    { \hspace*{\fill} \\}
    
    \begin{Verbatim}[commandchars=\\\{\}]
{\color{incolor}In [{\color{incolor}142}]:} \PY{n}{fig2}\PY{o}{=} \PY{n}{plt}\PY{o}{.}\PY{n}{figure}\PY{p}{(}\PY{l+m+mi}{2}\PY{p}{,} \PY{n}{figsize}\PY{o}{=} \PY{n}{size}\PY{p}{)}
          \PY{n}{axes2}\PY{o}{=} \PY{n}{fig2}\PY{o}{.}\PY{n}{add\PYZus{}subplot}\PY{p}{(}\PY{l+m+mi}{111}\PY{p}{)}
          \PY{n}{axes2}\PY{o}{.}\PY{n}{grid}\PY{p}{(}\PY{n+nb+bp}{True}\PY{p}{)}
          \PY{n}{axes2}\PY{o}{.}\PY{n}{set\PYZus{}xlabel}\PY{p}{(}\PY{l+s+s2}{\PYZdq{}}\PY{l+s+s2}{\PYZdl{}x\PYZdl{}}\PY{l+s+s2}{\PYZdq{}}\PY{p}{)}
          \PY{n}{axes2}\PY{o}{.}\PY{n}{set\PYZus{}title}\PY{p}{(}\PY{l+s+s2}{\PYZdq{}}\PY{l+s+s2}{\PYZdl{}cos(cos(x))}\PY{l+s+s2}{\PYZdq{}}\PY{p}{)}
          \PY{n}{axes2}\PY{o}{.}\PY{n}{set\PYZus{}ylabel}\PY{p}{(}\PY{l+s+s2}{\PYZdq{}}\PY{l+s+s2}{\PYZdl{}cos(cos(x))\PYZdl{}}\PY{l+s+s2}{\PYZdq{}}\PY{p}{)}
          \PY{n}{graph}\PY{o}{=} \PY{n}{axes2}\PY{o}{.}\PY{n}{plot}\PY{p}{(}\PY{n}{valuerange}\PY{p}{,} \PY{n}{coscos}\PY{p}{(}\PY{n}{valuerange}\PY{p}{)}\PY{p}{,}\PY{l+s+s2}{\PYZdq{}}\PY{l+s+s2}{bo}\PY{l+s+s2}{\PYZdq{}}\PY{p}{)}
\end{Verbatim}


    \begin{center}
    \adjustimage{max size={0.9\linewidth}{0.9\paperheight}}{output_7_0.png}
    \end{center}
    { \hspace*{\fill} \\}
    \begin{multicols}{2}
    As mentioned, these functions are extended periodically over the real
line with a period of $2\pi$. Thus we compute their first 51 fourier
coefficents as follows
    


\begin{align}
a_0 &= \frac{1}{2\pi}\int^{2\pi}_0 f(x)dx\\
a_n &= \frac{1}{\pi}\int^{2\pi}_0 f(x)cos(nx)dx\\
b_n &= \frac{1}{\pi}\int^{2\pi}_0 f(x)sin(nx)dx
\end{align}


We make use of the $quad$ function to determine the first 25
coeffiecients using a for loop. These coefficients are computed using a
for loop for both the functions and then plotted on a semilog and loglog
scale.

    \end{multicols}


    \begin{Verbatim}[commandchars=\\\{\}]
{\color{incolor}In [{\color{incolor}143}]:} \PY{c+c1}{\PYZsh{} to find the vector of 25 pairs of coefficients}
          
          \PY{c+c1}{\PYZsh{}find the coefficients for exponential}
          
          \PY{n}{coffexp}\PY{o}{=}\PY{p}{[}\PY{p}{]}
          \PY{n}{a0}\PY{o}{=} \PY{n}{quad}\PY{p}{(}\PY{n}{exponential}\PY{p}{,} \PY{l+m+mi}{0}\PY{p}{,} \PY{l+m+mi}{2}\PY{o}{*}\PY{n}{np}\PY{o}{.}\PY{n}{pi}\PY{p}{)}\PY{p}{[}\PY{l+m+mi}{0}\PY{p}{]}
          
          \PY{n}{a0}\PY{o}{=} \PY{n}{a0}\PY{o}{/}\PY{p}{(}\PY{l+m+mi}{2}\PY{p}{)}
          
          \PY{n}{coffexp}\PY{o}{.}\PY{n}{append}\PY{p}{(}\PY{n}{a0}\PY{p}{)}
          \PY{k}{for} \PY{n}{i} \PY{o+ow}{in} \PY{n+nb}{range}\PY{p}{(}\PY{l+m+mi}{1}\PY{p}{,}\PY{l+m+mi}{26}\PY{p}{)}\PY{p}{:}
              \PY{n}{ai}\PY{o}{=} \PY{n}{quad}\PY{p}{(}\PY{n}{expcos}\PY{p}{,} \PY{l+m+mi}{0}\PY{p}{,} \PY{l+m+mi}{2}\PY{o}{*}\PY{n}{np}\PY{o}{.}\PY{n}{pi}\PY{p}{,} \PY{n}{args}\PY{o}{=}\PY{p}{(}\PY{n}{i}\PY{p}{)}\PY{p}{)}\PY{p}{[}\PY{l+m+mi}{0}\PY{p}{]}
              \PY{n}{bi}\PY{o}{=} \PY{n}{quad}\PY{p}{(}\PY{n}{expsin}\PY{p}{,} \PY{l+m+mi}{0}\PY{p}{,} \PY{l+m+mi}{2}\PY{o}{*}\PY{n}{np}\PY{o}{.}\PY{n}{pi}\PY{p}{,} \PY{n}{args}\PY{o}{=}\PY{p}{(}\PY{n}{i}\PY{p}{)}\PY{p}{)}\PY{p}{[}\PY{l+m+mi}{0}\PY{p}{]}
              \PY{n}{coffexp}\PY{o}{.}\PY{n}{append}\PY{p}{(}\PY{n}{ai}\PY{p}{)}
              \PY{n}{coffexp}\PY{o}{.}\PY{n}{append}\PY{p}{(}\PY{n}{bi}\PY{p}{)}
          
          \PY{n}{coffexp}\PY{o}{=} \PY{p}{(}\PY{n}{np}\PY{o}{.}\PY{n}{array}\PY{p}{(}\PY{n}{coffexp}\PY{p}{)}\PY{p}{)}\PY{o}{/}\PY{p}{(}\PY{n}{np}\PY{o}{.}\PY{n}{pi}\PY{p}{)}
          
          
          \PY{n}{fig3}\PY{o}{=} \PY{n}{plt}\PY{o}{.}\PY{n}{figure}\PY{p}{(}\PY{l+m+mi}{3}\PY{p}{,} \PY{n}{figsize}\PY{o}{=} \PY{n}{size}\PY{p}{)}
          \PY{n}{axes30}\PY{o}{=} \PY{n}{fig3}\PY{o}{.}\PY{n}{add\PYZus{}subplot}\PY{p}{(}\PY{l+m+mi}{211}\PY{p}{)}
          \PY{n}{axes30}\PY{o}{.}\PY{n}{set\PYZus{}xlabel}\PY{p}{(}\PY{l+s+s2}{\PYZdq{}}\PY{l+s+s2}{x}\PY{l+s+s2}{\PYZdq{}}\PY{p}{)}
          \PY{n}{axes30}\PY{o}{.}\PY{n}{set\PYZus{}ylabel}\PY{p}{(}\PY{l+s+s2}{\PYZdq{}}\PY{l+s+s2}{Coefficients for \PYZdl{}e\PYZca{}x\PYZdl{}}\PY{l+s+s2}{\PYZdq{}}\PY{p}{)}
          \PY{n}{axes30}\PY{o}{.}\PY{n}{set\PYZus{}title}\PY{p}{(}\PY{l+s+s2}{\PYZdq{}}\PY{l+s+s2}{Fourier series coefficients on semilog scale}\PY{l+s+s2}{\PYZdq{}}\PY{p}{)}
          \PY{n}{axes30}\PY{o}{.}\PY{n}{grid}\PY{p}{(}\PY{n+nb+bp}{True}\PY{p}{)}
          \PY{n}{graph}\PY{o}{=} \PY{n}{axes30}\PY{o}{.}\PY{n}{semilogy}\PY{p}{(} \PY{n+nb}{abs}\PY{p}{(}\PY{n}{coffexp}\PY{p}{)}\PY{p}{,} \PY{l+s+s2}{\PYZdq{}}\PY{l+s+s2}{ro}\PY{l+s+s2}{\PYZdq{}}\PY{p}{)}
          
          \PY{n}{axes31}\PY{o}{=} \PY{n}{fig3}\PY{o}{.}\PY{n}{add\PYZus{}subplot}\PY{p}{(}\PY{l+m+mi}{212}\PY{p}{)}
          \PY{n}{axes31}\PY{o}{.}\PY{n}{set\PYZus{}xlabel}\PY{p}{(}\PY{l+s+s2}{\PYZdq{}}\PY{l+s+s2}{x}\PY{l+s+s2}{\PYZdq{}}\PY{p}{)}
          \PY{n}{axes31}\PY{o}{.}\PY{n}{set\PYZus{}ylabel}\PY{p}{(}\PY{l+s+s2}{\PYZdq{}}\PY{l+s+s2}{Coefficients for \PYZdl{}e\PYZca{}x\PYZdl{}}\PY{l+s+s2}{\PYZdq{}}\PY{p}{)}
          \PY{n}{axes31}\PY{o}{.}\PY{n}{set\PYZus{}title}\PY{p}{(}\PY{l+s+s2}{\PYZdq{}}\PY{l+s+s2}{Fourier series coefficients on loglog scale}\PY{l+s+s2}{\PYZdq{}}\PY{p}{)}
          \PY{n}{axes31}\PY{o}{.}\PY{n}{grid}\PY{p}{(}\PY{n+nb+bp}{True}\PY{p}{)}
          \PY{n}{graph}\PY{o}{=} \PY{n}{axes31}\PY{o}{.}\PY{n}{loglog}\PY{p}{(} \PY{n+nb}{abs}\PY{p}{(}\PY{n}{coffexp}\PY{p}{)}\PY{p}{,} \PY{l+s+s2}{\PYZdq{}}\PY{l+s+s2}{ro}\PY{l+s+s2}{\PYZdq{}}\PY{p}{)}
          \PY{n}{plt}\PY{o}{.}\PY{n}{tight\PYZus{}layout}\PY{p}{(}\PY{p}{)}
\end{Verbatim}


    \begin{center}
    \adjustimage{max size={0.9\linewidth}{0.9\paperheight}}{output_9_0.png}
    \end{center}
    { \hspace*{\fill} \\}
    
    \begin{Verbatim}[commandchars=\\\{\}]
{\color{incolor}In [{\color{incolor}144}]:} \PY{c+c1}{\PYZsh{} to find the vector of 25 coefficients}
          
          \PY{c+c1}{\PYZsh{}find the coefficients for exponential}
          
          \PY{n}{coffcc}\PY{o}{=}\PY{p}{[}\PY{p}{]}
          \PY{n}{a0}\PY{o}{=} \PY{n}{quad}\PY{p}{(}\PY{n}{coscos}\PY{p}{,} \PY{l+m+mi}{0}\PY{p}{,} \PY{l+m+mi}{2}\PY{o}{*}\PY{n}{np}\PY{o}{.}\PY{n}{pi}\PY{p}{)}\PY{p}{[}\PY{l+m+mi}{0}\PY{p}{]}
          
          \PY{n}{a0}\PY{o}{=} \PY{n}{a0}\PY{o}{/}\PY{p}{(}\PY{l+m+mi}{2}\PY{p}{)}
          \PY{n}{coffcc}\PY{o}{.}\PY{n}{append}\PY{p}{(}\PY{n}{a0}\PY{p}{)}
          \PY{k}{for} \PY{n}{i} \PY{o+ow}{in} \PY{n+nb}{range}\PY{p}{(}\PY{l+m+mi}{1}\PY{p}{,}\PY{l+m+mi}{26}\PY{p}{)}\PY{p}{:}
              \PY{n}{ai}\PY{o}{=} \PY{n}{quad}\PY{p}{(}\PY{n}{cccos}\PY{p}{,} \PY{l+m+mi}{0}\PY{p}{,} \PY{l+m+mi}{2}\PY{o}{*}\PY{n}{np}\PY{o}{.}\PY{n}{pi}\PY{p}{,} \PY{n}{args}\PY{o}{=}\PY{p}{(}\PY{n}{i}\PY{p}{)}\PY{p}{)}\PY{p}{[}\PY{l+m+mi}{0}\PY{p}{]}
              \PY{n}{bi}\PY{o}{=} \PY{n}{quad}\PY{p}{(}\PY{n}{ccsin}\PY{p}{,} \PY{l+m+mi}{0}\PY{p}{,} \PY{l+m+mi}{2}\PY{o}{*}\PY{n}{np}\PY{o}{.}\PY{n}{pi}\PY{p}{,} \PY{n}{args}\PY{o}{=}\PY{p}{(}\PY{n}{i}\PY{p}{)}\PY{p}{)}\PY{p}{[}\PY{l+m+mi}{0}\PY{p}{]}
              \PY{n}{coffcc}\PY{o}{.}\PY{n}{append}\PY{p}{(}\PY{n}{ai}\PY{p}{)}
              \PY{n}{coffcc}\PY{o}{.}\PY{n}{append}\PY{p}{(}\PY{n}{bi}\PY{p}{)}
          
          \PY{n}{coffcc}\PY{o}{=} \PY{p}{(}\PY{n}{np}\PY{o}{.}\PY{n}{array}\PY{p}{(}\PY{n}{coffcc}\PY{p}{)}\PY{p}{)}\PY{o}{/}\PY{p}{(}\PY{n}{np}\PY{o}{.}\PY{n}{pi}\PY{p}{)}
          
          
          \PY{n}{fig4}\PY{o}{=} \PY{n}{plt}\PY{o}{.}\PY{n}{figure}\PY{p}{(}\PY{l+m+mi}{4}\PY{p}{,} \PY{n}{figsize}\PY{o}{=} \PY{n}{size}\PY{p}{)}
          \PY{n}{axes40}\PY{o}{=} \PY{n}{fig4}\PY{o}{.}\PY{n}{add\PYZus{}subplot}\PY{p}{(}\PY{l+m+mi}{2}\PY{p}{,}\PY{l+m+mi}{1}\PY{p}{,}\PY{l+m+mi}{1}\PY{p}{)}
          \PY{n}{axes40}\PY{o}{.}\PY{n}{set\PYZus{}xlabel}\PY{p}{(}\PY{l+s+s2}{\PYZdq{}}\PY{l+s+s2}{x}\PY{l+s+s2}{\PYZdq{}}\PY{p}{)}
          \PY{n}{axes40}\PY{o}{.}\PY{n}{set\PYZus{}ylabel}\PY{p}{(}\PY{l+s+s2}{\PYZdq{}}\PY{l+s+s2}{Coefficients for \PYZdl{}cos(cos(x))\PYZdl{}}\PY{l+s+s2}{\PYZdq{}}\PY{p}{)}
          \PY{n}{axes40}\PY{o}{.}\PY{n}{set\PYZus{}title}\PY{p}{(}\PY{l+s+s2}{\PYZdq{}}\PY{l+s+s2}{Fourier series coefficients on semilog scale}\PY{l+s+s2}{\PYZdq{}}\PY{p}{)}
          \PY{n}{axes40}\PY{o}{.}\PY{n}{grid}\PY{p}{(}\PY{n+nb+bp}{True}\PY{p}{)}
          \PY{n}{graph}\PY{o}{=} \PY{n}{axes40}\PY{o}{.}\PY{n}{semilogy}\PY{p}{(} \PY{n+nb}{abs}\PY{p}{(}\PY{n}{coffcc}\PY{p}{)}\PY{p}{,} \PY{l+s+s2}{\PYZdq{}}\PY{l+s+s2}{ro}\PY{l+s+s2}{\PYZdq{}}\PY{p}{)}
          
          \PY{n}{axes41}\PY{o}{=} \PY{n}{fig4}\PY{o}{.}\PY{n}{add\PYZus{}subplot}\PY{p}{(}\PY{l+m+mi}{2}\PY{p}{,}\PY{l+m+mi}{1}\PY{p}{,}\PY{l+m+mi}{2}\PY{p}{)}
          \PY{n}{axes41}\PY{o}{.}\PY{n}{loglog}\PY{p}{(} \PY{n+nb}{abs}\PY{p}{(}\PY{n}{coffcc}\PY{p}{)}\PY{p}{,} \PY{l+s+s2}{\PYZdq{}}\PY{l+s+s2}{ro}\PY{l+s+s2}{\PYZdq{}}\PY{p}{)}
          \PY{n}{axes41}\PY{o}{.}\PY{n}{set\PYZus{}xlabel}\PY{p}{(}\PY{l+s+s2}{\PYZdq{}}\PY{l+s+s2}{x}\PY{l+s+s2}{\PYZdq{}}\PY{p}{)}
          \PY{n}{axes41}\PY{o}{.}\PY{n}{set\PYZus{}ylabel}\PY{p}{(}\PY{l+s+s2}{\PYZdq{}}\PY{l+s+s2}{Coefficients for \PYZdl{}cos(cos(x))\PYZdl{}}\PY{l+s+s2}{\PYZdq{}}\PY{p}{)}
          \PY{n}{axes41}\PY{o}{.}\PY{n}{grid}\PY{p}{(}\PY{n+nb+bp}{True}\PY{p}{)}
          \PY{n}{axes41}\PY{o}{.}\PY{n}{set\PYZus{}title}\PY{p}{(}\PY{l+s+s2}{\PYZdq{}}\PY{l+s+s2}{Fourier series coefficients on loglog scale}\PY{l+s+s2}{\PYZdq{}}\PY{p}{)}
          
          \PY{n}{plt}\PY{o}{.}\PY{n}{tight\PYZus{}layout}\PY{p}{(}\PY{p}{)}
\end{Verbatim}


    \begin{center}
    \adjustimage{max size={0.9\linewidth}{0.9\paperheight}}{output_10_0.png}
    \end{center}
    { \hspace*{\fill} \\}
    \begin{multicols}{2}
    The least squares approach is an alternate approach to direct
integration. Here we try to approximate the given functions $f_1$ and
$f_2$ with only 25 sinusoids. The vector $x$ in the following code
contains 400 equally spaced values from $0$ to $2\pi$ (inclusive). For a
given function, the following code constructs a vector $b$ of the values
of the function at each of the values in $x$. The vector A is the
matrix  \\


A=
$\left( \begin{matrix} 1 & cosx_1 & sinx_1 & ... & cos25x_1 & sin25x_1\\ 1 & cosx_2 & sinx_2 & ... & cos25x_2 & sin25x_2\\ ... & ... & ... & ... & ..._1 & ...\\ 1 & cosx_{400} & sinx_{400} & ... & cos25x_{400} & sin25x_{400} \end{matrix} \right)$ \\



The piece of code $c= np.linalg.lstsq(A,b)[0]$ determines a vector $c$
that best suits the equation
\begin{equation}
Ac = b.
\end{equation}

\end{multicols}

    \begin{Verbatim}[commandchars=\\\{\}]
{\color{incolor}In [{\color{incolor}145}]:} \PY{n}{x}\PY{o}{=} \PY{n}{np}\PY{o}{.}\PY{n}{linspace}\PY{p}{(}\PY{l+m+mi}{0}\PY{p}{,} \PY{l+m+mi}{2}\PY{o}{*}\PY{n}{np}\PY{o}{.}\PY{n}{pi}\PY{p}{,} \PY{l+m+mi}{401}\PY{p}{)}
          \PY{n}{x}\PY{o}{=} \PY{n}{x}\PY{p}{[}\PY{p}{:}\PY{o}{\PYZhy{}}\PY{l+m+mi}{1}\PY{p}{]}
          \PY{n}{A}\PY{o}{=} \PY{n}{np}\PY{o}{.}\PY{n}{zeros}\PY{p}{(}\PY{p}{(}\PY{l+m+mi}{400}\PY{p}{,}\PY{l+m+mi}{51}\PY{p}{)}\PY{p}{)}
          \PY{n}{A}\PY{p}{[}\PY{p}{:}\PY{p}{,}\PY{l+m+mi}{0}\PY{p}{]}\PY{o}{=} \PY{l+m+mi}{1}
          \PY{k}{for} \PY{n}{i} \PY{o+ow}{in} \PY{n+nb}{range}\PY{p}{(}\PY{l+m+mi}{1}\PY{p}{,}\PY{l+m+mi}{26}\PY{p}{)}\PY{p}{:}
              \PY{n}{A}\PY{p}{[}\PY{p}{:}\PY{p}{,}\PY{l+m+mi}{2}\PY{o}{*}\PY{n}{i} \PY{o}{\PYZhy{}}\PY{l+m+mi}{1}\PY{p}{]}\PY{o}{=} \PY{n}{np}\PY{o}{.}\PY{n}{cos}\PY{p}{(}\PY{n}{i}\PY{o}{*}\PY{n}{x}\PY{p}{)}
              \PY{n}{A}\PY{p}{[}\PY{p}{:}\PY{p}{,}\PY{l+m+mi}{2}\PY{o}{*}\PY{n}{i}\PY{p}{]}\PY{o}{=} \PY{n}{np}\PY{o}{.}\PY{n}{sin}\PY{p}{(}\PY{n}{i}\PY{o}{*}\PY{n}{x}\PY{p}{)}
              
          \PY{n}{b\PYZus{}exp}\PY{o}{=} \PY{n}{exponential}\PY{p}{(}\PY{n}{x}\PY{p}{)}
          \PY{n}{b\PYZus{}cc}\PY{o}{=} \PY{n}{coscos}\PY{p}{(}\PY{n}{x}\PY{p}{)}
          
          \PY{n}{c\PYZus{}exp}\PY{o}{=} \PY{n}{np}\PY{o}{.}\PY{n}{linalg}\PY{o}{.}\PY{n}{lstsq}\PY{p}{(}\PY{n}{A}\PY{p}{,}\PY{n}{b\PYZus{}exp}\PY{p}{)}\PY{p}{[}\PY{l+m+mi}{0}\PY{p}{]}
          \PY{n}{c\PYZus{}cc}\PY{o}{=} \PY{n}{np}\PY{o}{.}\PY{n}{linalg}\PY{o}{.}\PY{n}{lstsq}\PY{p}{(}\PY{n}{A}\PY{p}{,}\PY{n}{b\PYZus{}cc}\PY{p}{)}\PY{p}{[}\PY{l+m+mi}{0}\PY{p}{]}
\end{Verbatim}

\begin{multicols}{2}

    The obtained coefficients are plotted on the corresponding figures to
compare the deviation with the coefficients obtained from the $quad$
function.

\end{multicols}


    \begin{Verbatim}[commandchars=\\\{\}]
{\color{incolor}In [{\color{incolor}146}]:} \PY{n}{graph}\PY{o}{=} \PY{n}{axes30}\PY{o}{.}\PY{n}{semilogy}\PY{p}{(}\PY{n+nb}{abs}\PY{p}{(}\PY{n}{c\PYZus{}exp}\PY{p}{)}\PY{p}{,} \PY{l+s+s2}{\PYZdq{}}\PY{l+s+s2}{go}\PY{l+s+s2}{\PYZdq{}}\PY{p}{)}
          \PY{n}{graph}\PY{o}{=} \PY{n}{axes31}\PY{o}{.}\PY{n}{loglog}\PY{p}{(}\PY{n+nb}{abs}\PY{p}{(}\PY{n}{c\PYZus{}exp}\PY{p}{)}\PY{p}{,} \PY{l+s+s2}{\PYZdq{}}\PY{l+s+s2}{go}\PY{l+s+s2}{\PYZdq{}}\PY{p}{)}
          \PY{n}{fig3}
\end{Verbatim}

\texttt{\color{outcolor}Out[{\color{outcolor}146}]:}
    
    \begin{center}
    \adjustimage{max size={0.9\linewidth}{0.9\paperheight}}{output_14_0.png}
    \end{center}
    { \hspace*{\fill} \\}
    

    \begin{Verbatim}[commandchars=\\\{\}]
{\color{incolor}In [{\color{incolor}147}]:} \PY{n}{graph}\PY{o}{=} \PY{n}{axes40}\PY{o}{.}\PY{n}{semilogy}\PY{p}{(}\PY{n+nb}{abs}\PY{p}{(}\PY{n}{c\PYZus{}cc}\PY{p}{)}\PY{p}{,} \PY{l+s+s2}{\PYZdq{}}\PY{l+s+s2}{go}\PY{l+s+s2}{\PYZdq{}}\PY{p}{)}
          \PY{n}{graph}\PY{o}{=} \PY{n}{axes41}\PY{o}{.}\PY{n}{loglog}\PY{p}{(}\PY{n+nb}{abs}\PY{p}{(}\PY{n}{c\PYZus{}cc}\PY{p}{)}\PY{p}{,} \PY{l+s+s2}{\PYZdq{}}\PY{l+s+s2}{go}\PY{l+s+s2}{\PYZdq{}}\PY{p}{)}
          \PY{n}{fig4}
\end{Verbatim}

\texttt{\color{outcolor}Out[{\color{outcolor}147}]:}
    
    \begin{center}
    \adjustimage{max size={0.9\linewidth}{0.9\paperheight}}{output_15_0.png}
    \end{center}
    { \hspace*{\fill} \\}
    
\begin{multicols}{2}
    The errors between the coefficients computed by quad and the least
squares approach are computed.
\end{multicols}

    \begin{Verbatim}[commandchars=\\\{\}]
{\color{incolor}In [{\color{incolor}148}]:} \PY{k}{print} \PY{n+nb}{max}\PY{p}{(}\PY{n+nb}{abs}\PY{p}{(}\PY{n}{c\PYZus{}exp}\PY{o}{\PYZhy{}}\PY{n}{coffexp}\PY{p}{)}\PY{p}{)}
          \PY{k}{print} \PY{n+nb}{max}\PY{p}{(}\PY{n+nb}{abs}\PY{p}{(}\PY{n}{c\PYZus{}cc}\PY{o}{\PYZhy{}}\PY{n}{coffcc}\PY{p}{)}\PY{p}{)}
\end{Verbatim}


    \begin{Verbatim}[commandchars=\\\{\}]
1.33273087034
2.57022928446e-15

    \end{Verbatim}
\begin{multicols}{2}
    The vector $Ac$ is computed to determine the value of the function at
the 400 points given by vector x. This is done for both the functions
and plotted on the correspoding graphs in green.
\end{multicols}

    \begin{Verbatim}[commandchars=\\\{\}]
{\color{incolor}In [{\color{incolor}149}]:} \PY{n}{expvalues}\PY{o}{=} \PY{n}{np}\PY{o}{.}\PY{n}{dot}\PY{p}{(}\PY{n}{A}\PY{p}{,}\PY{n}{c\PYZus{}exp}\PY{p}{)}
          
          \PY{n}{graph}\PY{o}{=} \PY{n}{axes1}\PY{o}{.}\PY{n}{semilogy}\PY{p}{(}\PY{n}{x}\PY{p}{,} \PY{n}{expvalues}\PY{p}{,} \PY{l+s+s1}{\PYZsq{}}\PY{l+s+s1}{go}\PY{l+s+s1}{\PYZsq{}}\PY{p}{)}
          \PY{n}{fig1}
\end{Verbatim}

\texttt{\color{outcolor}Out[{\color{outcolor}149}]:}
    
    \begin{center}
    \adjustimage{max size={0.9\linewidth}{0.9\paperheight}}{output_19_0.png}
    \end{center}
    { \hspace*{\fill} \\}
    

    \begin{Verbatim}[commandchars=\\\{\}]
{\color{incolor}In [{\color{incolor}150}]:} \PY{n}{ccvalues}\PY{o}{=} \PY{n}{np}\PY{o}{.}\PY{n}{dot}\PY{p}{(}\PY{n}{A}\PY{p}{,}\PY{n}{c\PYZus{}cc}\PY{p}{)}
          \PY{n}{graph}\PY{o}{=}\PY{n}{axes2}\PY{o}{.}\PY{n}{plot}\PY{p}{(}\PY{n}{x}\PY{p}{,} \PY{n}{ccvalues}\PY{p}{,} \PY{l+s+s2}{\PYZdq{}}\PY{l+s+s2}{go}\PY{l+s+s2}{\PYZdq{}}\PY{p}{)}
          \PY{n}{fig2}
\end{Verbatim}

\texttt{\color{outcolor}Out[{\color{outcolor}150}]:}
    
    \begin{center}
    \adjustimage{max size={0.9\linewidth}{0.9\paperheight}}{output_20_0.png}
    \end{center}
    { \hspace*{\fill} \\}
    
\begin{multicols}{2}
    \section{Discussion and conclusion}\label{discussion-and-conlusion}

\paragraph{} From figure1 and figure2 we note that the function $e^x$ is clearly
aperodic and the function $cos(cos(x))$ is periodic with a fundamental
period $\pi$. The fourier series with the coefficients computed will
result in $2\pi$ periodic function that is extended to the real line.
Further upon computing and plotting the fourier coefficients, the $b_n$
coefficeints for the $cos(cos(x)$ function turn out to be zero. This, of
course is a consequence of the function being even, which can
alternately be understood as symmetry about the line $x = \pi$. 

\paragraph{} The fourier coefficients do not converge as quickly for $e^x$ as they do for
$cos(cos(x)$. This is because the latter function is infinitely
diffrentiable and thus its fourier coefficients experience exponential
decay. The exponential function, extended the way it is, has jump
discontinuities and thus, its $k^th$ coefficients decay as $\frac{1}{k}$
which is slower than exponential decay. Thus in the case of $e^x$ the
loglog plot looks linear as its coefficients decay as $\frac{1}{k}$.

\begin{align*}
a_k  &\propto  \frac{1}{k^r}\\
log(a_k)  &\propto -log(k)
\end{align*}

However in the case of $cos(cos(x))$, the coefficients decay as $r^k$

\begin{align*}
a_k  &\propto  r^k,   &r<1\\
log(a_k)  &\propto  klog(r),  &log(r)<0
\end{align*}

This results in the semilog plot being a straight line with a negative
slope.
\paragraph{} From the replotted figures, it is evident that the fourier coefficients
agree closely for the $cos(cos(x))$ function. This follows from the
above argument. Rapid decay of the coefficients implies that the first
few coefficients are sufficient to "fit" or describe the function well
and hence will closely agree with the result of the least squares "best
fit". The same can not be said of the coefficients of the exponential
function which decay much slower. Hence, the large deviation in
coefficients.
\paragraph{} This also explains why the recomputed function ($Ac$) will
not agree closely with the original function in case of $e^x$. This plot
displays ripples due to Gibbs phenomenon which occurs due to the
presence of a jump discontinuity in the function and the use of a finite
sequence approximation.
\paragraph{} Thus, this work makes a study of the variation of the fourier
coefficients and the rate of convergence of the series for different
functions. It does so by making comprehensive use of scientific python
to determine the coefficients in multiple ways, plot them and compare
them.

\end{multicols}
    % Add a bibliography block to the postdoc
    
    
    
    \end{document}
