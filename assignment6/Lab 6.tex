
% Default to the notebook output style

    


% Inherit from the specified cell style.




    
\documentclass[a4paper, 11pt, margin= 1.25cm]{article}

    
    \usepackage{multicol}
    \usepackage[T1]{fontenc}
    % Nicer default font (+ math font) than Computer Modern for most use cases
    \usepackage{mathpazo}

    % Basic figure setup, for now with no caption control since it's done
    % automatically by Pandoc (which extracts ![](path) syntax from Markdown).
    \usepackage{graphicx}
    % We will generate all images so they have a width \maxwidth. This means
    % that they will get their normal width if they fit onto the page, but
    % are scaled down if they would overflow the margins.
    \makeatletter
    \def\maxwidth{\ifdim\Gin@nat@width>\linewidth\linewidth
    \else\Gin@nat@width\fi}
    \makeatother
    \let\Oldincludegraphics\includegraphics
    % Set max figure width to be 80% of text width, for now hardcoded.
    \renewcommand{\includegraphics}[1]{\Oldincludegraphics[width=.8\maxwidth]{#1}}
    % Ensure that by default, figures have no caption (until we provide a
    % proper Figure object with a Caption API and a way to capture that
    % in the conversion process - todo).
    \usepackage{caption}
    \DeclareCaptionLabelFormat{nolabel}{}
    \captionsetup{labelformat=nolabel}

    \usepackage{adjustbox} % Used to constrain images to a maximum size 
    \usepackage{xcolor} % Allow colors to be defined
    \usepackage{enumerate} % Needed for markdown enumerations to work
    \usepackage{geometry} % Used to adjust the document margins
    \usepackage{amsmath} % Equations
    \usepackage{amssymb} % Equations
    \usepackage{textcomp} % defines textquotesingle
    % Hack from http://tex.stackexchange.com/a/47451/13684:
    \AtBeginDocument{%
        \def\PYZsq{\textquotesingle}% Upright quotes in Pygmentized code
    }
    \usepackage{upquote} % Upright quotes for verbatim code
    \usepackage{eurosym} % defines \euro
    \usepackage[mathletters]{ucs} % Extended unicode (utf-8) support
    \usepackage[utf8x]{inputenc} % Allow utf-8 characters in the tex document
    \usepackage{fancyvrb} % verbatim replacement that allows latex
    \usepackage{grffile} % extends the file name processing of package graphics 
                         % to support a larger range 
    % The hyperref package gives us a pdf with properly built
    % internal navigation ('pdf bookmarks' for the table of contents,
    % internal cross-reference links, web links for URLs, etc.)
    \usepackage{hyperref}
    \usepackage{longtable} % longtable support required by pandoc >1.10
    \usepackage{booktabs}  % table support for pandoc > 1.12.2
    \usepackage[inline]{enumitem} % IRkernel/repr support (it uses the enumerate* environment)
    \usepackage[normalem]{ulem} % ulem is needed to support strikethroughs (\sout)
                                % normalem makes italics be italics, not underlines
    

    
    
    % Colors for the hyperref package
    \definecolor{urlcolor}{rgb}{0,.145,.698}
    \definecolor{linkcolor}{rgb}{.71,0.21,0.01}
    \definecolor{citecolor}{rgb}{.12,.54,.11}

    % ANSI colors
    \definecolor{ansi-black}{HTML}{3E424D}
    \definecolor{ansi-black-intense}{HTML}{282C36}
    \definecolor{ansi-red}{HTML}{E75C58}
    \definecolor{ansi-red-intense}{HTML}{B22B31}
    \definecolor{ansi-green}{HTML}{00A250}
    \definecolor{ansi-green-intense}{HTML}{007427}
    \definecolor{ansi-yellow}{HTML}{DDB62B}
    \definecolor{ansi-yellow-intense}{HTML}{B27D12}
    \definecolor{ansi-blue}{HTML}{208FFB}
    \definecolor{ansi-blue-intense}{HTML}{0065CA}
    \definecolor{ansi-magenta}{HTML}{D160C4}
    \definecolor{ansi-magenta-intense}{HTML}{A03196}
    \definecolor{ansi-cyan}{HTML}{60C6C8}
    \definecolor{ansi-cyan-intense}{HTML}{258F8F}
    \definecolor{ansi-white}{HTML}{C5C1B4}
    \definecolor{ansi-white-intense}{HTML}{A1A6B2}

    % commands and environments needed by pandoc snippets
    % extracted from the output of `pandoc -s`
    \providecommand{\tightlist}{%
      \setlength{\itemsep}{0pt}\setlength{\parskip}{0pt}}
    \DefineVerbatimEnvironment{Highlighting}{Verbatim}{commandchars=\\\{\}}
    % Add ',fontsize=\small' for more characters per line
    \newenvironment{Shaded}{}{}
    \newcommand{\KeywordTok}[1]{\textcolor[rgb]{0.00,0.44,0.13}{\textbf{{#1}}}}
    \newcommand{\DataTypeTok}[1]{\textcolor[rgb]{0.56,0.13,0.00}{{#1}}}
    \newcommand{\DecValTok}[1]{\textcolor[rgb]{0.25,0.63,0.44}{{#1}}}
    \newcommand{\BaseNTok}[1]{\textcolor[rgb]{0.25,0.63,0.44}{{#1}}}
    \newcommand{\FloatTok}[1]{\textcolor[rgb]{0.25,0.63,0.44}{{#1}}}
    \newcommand{\CharTok}[1]{\textcolor[rgb]{0.25,0.44,0.63}{{#1}}}
    \newcommand{\StringTok}[1]{\textcolor[rgb]{0.25,0.44,0.63}{{#1}}}
    \newcommand{\CommentTok}[1]{\textcolor[rgb]{0.38,0.63,0.69}{\textit{{#1}}}}
    \newcommand{\OtherTok}[1]{\textcolor[rgb]{0.00,0.44,0.13}{{#1}}}
    \newcommand{\AlertTok}[1]{\textcolor[rgb]{1.00,0.00,0.00}{\textbf{{#1}}}}
    \newcommand{\FunctionTok}[1]{\textcolor[rgb]{0.02,0.16,0.49}{{#1}}}
    \newcommand{\RegionMarkerTok}[1]{{#1}}
    \newcommand{\ErrorTok}[1]{\textcolor[rgb]{1.00,0.00,0.00}{\textbf{{#1}}}}
    \newcommand{\NormalTok}[1]{{#1}}
    
    % Additional commands for more recent versions of Pandoc
    \newcommand{\ConstantTok}[1]{\textcolor[rgb]{0.53,0.00,0.00}{{#1}}}
    \newcommand{\SpecialCharTok}[1]{\textcolor[rgb]{0.25,0.44,0.63}{{#1}}}
    \newcommand{\VerbatimStringTok}[1]{\textcolor[rgb]{0.25,0.44,0.63}{{#1}}}
    \newcommand{\SpecialStringTok}[1]{\textcolor[rgb]{0.73,0.40,0.53}{{#1}}}
    \newcommand{\ImportTok}[1]{{#1}}
    \newcommand{\DocumentationTok}[1]{\textcolor[rgb]{0.73,0.13,0.13}{\textit{{#1}}}}
    \newcommand{\AnnotationTok}[1]{\textcolor[rgb]{0.38,0.63,0.69}{\textbf{\textit{{#1}}}}}
    \newcommand{\CommentVarTok}[1]{\textcolor[rgb]{0.38,0.63,0.69}{\textbf{\textit{{#1}}}}}
    \newcommand{\VariableTok}[1]{\textcolor[rgb]{0.10,0.09,0.49}{{#1}}}
    \newcommand{\ControlFlowTok}[1]{\textcolor[rgb]{0.00,0.44,0.13}{\textbf{{#1}}}}
    \newcommand{\OperatorTok}[1]{\textcolor[rgb]{0.40,0.40,0.40}{{#1}}}
    \newcommand{\BuiltInTok}[1]{{#1}}
    \newcommand{\ExtensionTok}[1]{{#1}}
    \newcommand{\PreprocessorTok}[1]{\textcolor[rgb]{0.74,0.48,0.00}{{#1}}}
    \newcommand{\AttributeTok}[1]{\textcolor[rgb]{0.49,0.56,0.16}{{#1}}}
    \newcommand{\InformationTok}[1]{\textcolor[rgb]{0.38,0.63,0.69}{\textbf{\textit{{#1}}}}}
    \newcommand{\WarningTok}[1]{\textcolor[rgb]{0.38,0.63,0.69}{\textbf{\textit{{#1}}}}}
    
    
    % Define a nice break command that doesn't care if a line doesn't already
    % exist.
    \def\br{\hspace*{\fill} \\* }
    % Math Jax compatability definitions
    \def\gt{>}
    \def\lt{<}
    % Document parameters
    \title{Modelling a tubelight}
    \date{11-03-2018}
    \author{Milind Kumar V\\ EE16B025}
    
    
    

    % Pygments definitions
    
\makeatletter
\def\PY@reset{\let\PY@it=\relax \let\PY@bf=\relax%
    \let\PY@ul=\relax \let\PY@tc=\relax%
    \let\PY@bc=\relax \let\PY@ff=\relax}
\def\PY@tok#1{\csname PY@tok@#1\endcsname}
\def\PY@toks#1+{\ifx\relax#1\empty\else%
    \PY@tok{#1}\expandafter\PY@toks\fi}
\def\PY@do#1{\PY@bc{\PY@tc{\PY@ul{%
    \PY@it{\PY@bf{\PY@ff{#1}}}}}}}
\def\PY#1#2{\PY@reset\PY@toks#1+\relax+\PY@do{#2}}

\expandafter\def\csname PY@tok@gd\endcsname{\def\PY@tc##1{\textcolor[rgb]{0.63,0.00,0.00}{##1}}}
\expandafter\def\csname PY@tok@gu\endcsname{\let\PY@bf=\textbf\def\PY@tc##1{\textcolor[rgb]{0.50,0.00,0.50}{##1}}}
\expandafter\def\csname PY@tok@gt\endcsname{\def\PY@tc##1{\textcolor[rgb]{0.00,0.27,0.87}{##1}}}
\expandafter\def\csname PY@tok@gs\endcsname{\let\PY@bf=\textbf}
\expandafter\def\csname PY@tok@gr\endcsname{\def\PY@tc##1{\textcolor[rgb]{1.00,0.00,0.00}{##1}}}
\expandafter\def\csname PY@tok@cm\endcsname{\let\PY@it=\textit\def\PY@tc##1{\textcolor[rgb]{0.25,0.50,0.50}{##1}}}
\expandafter\def\csname PY@tok@vg\endcsname{\def\PY@tc##1{\textcolor[rgb]{0.10,0.09,0.49}{##1}}}
\expandafter\def\csname PY@tok@vi\endcsname{\def\PY@tc##1{\textcolor[rgb]{0.10,0.09,0.49}{##1}}}
\expandafter\def\csname PY@tok@vm\endcsname{\def\PY@tc##1{\textcolor[rgb]{0.10,0.09,0.49}{##1}}}
\expandafter\def\csname PY@tok@mh\endcsname{\def\PY@tc##1{\textcolor[rgb]{0.40,0.40,0.40}{##1}}}
\expandafter\def\csname PY@tok@cs\endcsname{\let\PY@it=\textit\def\PY@tc##1{\textcolor[rgb]{0.25,0.50,0.50}{##1}}}
\expandafter\def\csname PY@tok@ge\endcsname{\let\PY@it=\textit}
\expandafter\def\csname PY@tok@vc\endcsname{\def\PY@tc##1{\textcolor[rgb]{0.10,0.09,0.49}{##1}}}
\expandafter\def\csname PY@tok@il\endcsname{\def\PY@tc##1{\textcolor[rgb]{0.40,0.40,0.40}{##1}}}
\expandafter\def\csname PY@tok@go\endcsname{\def\PY@tc##1{\textcolor[rgb]{0.53,0.53,0.53}{##1}}}
\expandafter\def\csname PY@tok@cp\endcsname{\def\PY@tc##1{\textcolor[rgb]{0.74,0.48,0.00}{##1}}}
\expandafter\def\csname PY@tok@gi\endcsname{\def\PY@tc##1{\textcolor[rgb]{0.00,0.63,0.00}{##1}}}
\expandafter\def\csname PY@tok@gh\endcsname{\let\PY@bf=\textbf\def\PY@tc##1{\textcolor[rgb]{0.00,0.00,0.50}{##1}}}
\expandafter\def\csname PY@tok@ni\endcsname{\let\PY@bf=\textbf\def\PY@tc##1{\textcolor[rgb]{0.60,0.60,0.60}{##1}}}
\expandafter\def\csname PY@tok@nl\endcsname{\def\PY@tc##1{\textcolor[rgb]{0.63,0.63,0.00}{##1}}}
\expandafter\def\csname PY@tok@nn\endcsname{\let\PY@bf=\textbf\def\PY@tc##1{\textcolor[rgb]{0.00,0.00,1.00}{##1}}}
\expandafter\def\csname PY@tok@no\endcsname{\def\PY@tc##1{\textcolor[rgb]{0.53,0.00,0.00}{##1}}}
\expandafter\def\csname PY@tok@na\endcsname{\def\PY@tc##1{\textcolor[rgb]{0.49,0.56,0.16}{##1}}}
\expandafter\def\csname PY@tok@nb\endcsname{\def\PY@tc##1{\textcolor[rgb]{0.00,0.50,0.00}{##1}}}
\expandafter\def\csname PY@tok@nc\endcsname{\let\PY@bf=\textbf\def\PY@tc##1{\textcolor[rgb]{0.00,0.00,1.00}{##1}}}
\expandafter\def\csname PY@tok@nd\endcsname{\def\PY@tc##1{\textcolor[rgb]{0.67,0.13,1.00}{##1}}}
\expandafter\def\csname PY@tok@ne\endcsname{\let\PY@bf=\textbf\def\PY@tc##1{\textcolor[rgb]{0.82,0.25,0.23}{##1}}}
\expandafter\def\csname PY@tok@nf\endcsname{\def\PY@tc##1{\textcolor[rgb]{0.00,0.00,1.00}{##1}}}
\expandafter\def\csname PY@tok@si\endcsname{\let\PY@bf=\textbf\def\PY@tc##1{\textcolor[rgb]{0.73,0.40,0.53}{##1}}}
\expandafter\def\csname PY@tok@s2\endcsname{\def\PY@tc##1{\textcolor[rgb]{0.73,0.13,0.13}{##1}}}
\expandafter\def\csname PY@tok@nt\endcsname{\let\PY@bf=\textbf\def\PY@tc##1{\textcolor[rgb]{0.00,0.50,0.00}{##1}}}
\expandafter\def\csname PY@tok@nv\endcsname{\def\PY@tc##1{\textcolor[rgb]{0.10,0.09,0.49}{##1}}}
\expandafter\def\csname PY@tok@s1\endcsname{\def\PY@tc##1{\textcolor[rgb]{0.73,0.13,0.13}{##1}}}
\expandafter\def\csname PY@tok@dl\endcsname{\def\PY@tc##1{\textcolor[rgb]{0.73,0.13,0.13}{##1}}}
\expandafter\def\csname PY@tok@ch\endcsname{\let\PY@it=\textit\def\PY@tc##1{\textcolor[rgb]{0.25,0.50,0.50}{##1}}}
\expandafter\def\csname PY@tok@m\endcsname{\def\PY@tc##1{\textcolor[rgb]{0.40,0.40,0.40}{##1}}}
\expandafter\def\csname PY@tok@gp\endcsname{\let\PY@bf=\textbf\def\PY@tc##1{\textcolor[rgb]{0.00,0.00,0.50}{##1}}}
\expandafter\def\csname PY@tok@sh\endcsname{\def\PY@tc##1{\textcolor[rgb]{0.73,0.13,0.13}{##1}}}
\expandafter\def\csname PY@tok@ow\endcsname{\let\PY@bf=\textbf\def\PY@tc##1{\textcolor[rgb]{0.67,0.13,1.00}{##1}}}
\expandafter\def\csname PY@tok@sx\endcsname{\def\PY@tc##1{\textcolor[rgb]{0.00,0.50,0.00}{##1}}}
\expandafter\def\csname PY@tok@bp\endcsname{\def\PY@tc##1{\textcolor[rgb]{0.00,0.50,0.00}{##1}}}
\expandafter\def\csname PY@tok@c1\endcsname{\let\PY@it=\textit\def\PY@tc##1{\textcolor[rgb]{0.25,0.50,0.50}{##1}}}
\expandafter\def\csname PY@tok@fm\endcsname{\def\PY@tc##1{\textcolor[rgb]{0.00,0.00,1.00}{##1}}}
\expandafter\def\csname PY@tok@o\endcsname{\def\PY@tc##1{\textcolor[rgb]{0.40,0.40,0.40}{##1}}}
\expandafter\def\csname PY@tok@kc\endcsname{\let\PY@bf=\textbf\def\PY@tc##1{\textcolor[rgb]{0.00,0.50,0.00}{##1}}}
\expandafter\def\csname PY@tok@c\endcsname{\let\PY@it=\textit\def\PY@tc##1{\textcolor[rgb]{0.25,0.50,0.50}{##1}}}
\expandafter\def\csname PY@tok@mf\endcsname{\def\PY@tc##1{\textcolor[rgb]{0.40,0.40,0.40}{##1}}}
\expandafter\def\csname PY@tok@err\endcsname{\def\PY@bc##1{\setlength{\fboxsep}{0pt}\fcolorbox[rgb]{1.00,0.00,0.00}{1,1,1}{\strut ##1}}}
\expandafter\def\csname PY@tok@mb\endcsname{\def\PY@tc##1{\textcolor[rgb]{0.40,0.40,0.40}{##1}}}
\expandafter\def\csname PY@tok@ss\endcsname{\def\PY@tc##1{\textcolor[rgb]{0.10,0.09,0.49}{##1}}}
\expandafter\def\csname PY@tok@sr\endcsname{\def\PY@tc##1{\textcolor[rgb]{0.73,0.40,0.53}{##1}}}
\expandafter\def\csname PY@tok@mo\endcsname{\def\PY@tc##1{\textcolor[rgb]{0.40,0.40,0.40}{##1}}}
\expandafter\def\csname PY@tok@kd\endcsname{\let\PY@bf=\textbf\def\PY@tc##1{\textcolor[rgb]{0.00,0.50,0.00}{##1}}}
\expandafter\def\csname PY@tok@mi\endcsname{\def\PY@tc##1{\textcolor[rgb]{0.40,0.40,0.40}{##1}}}
\expandafter\def\csname PY@tok@kn\endcsname{\let\PY@bf=\textbf\def\PY@tc##1{\textcolor[rgb]{0.00,0.50,0.00}{##1}}}
\expandafter\def\csname PY@tok@cpf\endcsname{\let\PY@it=\textit\def\PY@tc##1{\textcolor[rgb]{0.25,0.50,0.50}{##1}}}
\expandafter\def\csname PY@tok@kr\endcsname{\let\PY@bf=\textbf\def\PY@tc##1{\textcolor[rgb]{0.00,0.50,0.00}{##1}}}
\expandafter\def\csname PY@tok@s\endcsname{\def\PY@tc##1{\textcolor[rgb]{0.73,0.13,0.13}{##1}}}
\expandafter\def\csname PY@tok@kp\endcsname{\def\PY@tc##1{\textcolor[rgb]{0.00,0.50,0.00}{##1}}}
\expandafter\def\csname PY@tok@w\endcsname{\def\PY@tc##1{\textcolor[rgb]{0.73,0.73,0.73}{##1}}}
\expandafter\def\csname PY@tok@kt\endcsname{\def\PY@tc##1{\textcolor[rgb]{0.69,0.00,0.25}{##1}}}
\expandafter\def\csname PY@tok@sc\endcsname{\def\PY@tc##1{\textcolor[rgb]{0.73,0.13,0.13}{##1}}}
\expandafter\def\csname PY@tok@sb\endcsname{\def\PY@tc##1{\textcolor[rgb]{0.73,0.13,0.13}{##1}}}
\expandafter\def\csname PY@tok@sa\endcsname{\def\PY@tc##1{\textcolor[rgb]{0.73,0.13,0.13}{##1}}}
\expandafter\def\csname PY@tok@k\endcsname{\let\PY@bf=\textbf\def\PY@tc##1{\textcolor[rgb]{0.00,0.50,0.00}{##1}}}
\expandafter\def\csname PY@tok@se\endcsname{\let\PY@bf=\textbf\def\PY@tc##1{\textcolor[rgb]{0.73,0.40,0.13}{##1}}}
\expandafter\def\csname PY@tok@sd\endcsname{\let\PY@it=\textit\def\PY@tc##1{\textcolor[rgb]{0.73,0.13,0.13}{##1}}}

\def\PYZbs{\char`\\}
\def\PYZus{\char`\_}
\def\PYZob{\char`\{}
\def\PYZcb{\char`\}}
\def\PYZca{\char`\^}
\def\PYZam{\char`\&}
\def\PYZlt{\char`\<}
\def\PYZgt{\char`\>}
\def\PYZsh{\char`\#}
\def\PYZpc{\char`\%}
\def\PYZdl{\char`\$}
\def\PYZhy{\char`\-}
\def\PYZsq{\char`\'}
\def\PYZdq{\char`\"}
\def\PYZti{\char`\~}
% for compatibility with earlier versions
\def\PYZat{@}
\def\PYZlb{[}
\def\PYZrb{]}
\makeatother


    % Exact colors from NB
    %\defi       0.0, 0.0, 0.5}
    \definecolor{outcolor}{rgb}{0.545, 0.0, 0.0}



    
    % Prevent overflowing lines due to hard-to-break entities
    \sloppy 
    % Setup hyperref package
    \hypersetup{
      breaklinks=true,  % so long urls are correctly broken across lines
      colorlinks=true,
      urlcolor=urlcolor,
      linkcolor=linkcolor,
      citecolor=citecolor,
      }
    % Slightly bigger margins than the latex defaults
    
    \geometry{verbose,tmargin=1in,bmargin=1in,lmargin=1in,rmargin=1in}
    
    

    \begin{document}
    
    
    \maketitle
    
    

\begin{multicols}{2}
    
    \begin{abstract}\label{abstract}

Tubelights work by the emission of electrons at the cathode that undergo
accelecration due to an applied electric field and excite atoms which
emit photons during subsequent relaxation. Thus, models of tubelights
can be built efficiently and accurately using Python with the aid of
libraries such as numpy and scipy that enable one to visualise the
working of this device using intensity plots and phase spaces.
\end{abstract}
\section{Introduction}\label{introduction}

The tubelight is modeled as a one dimensional two terminal device with
the ends being the cathode and the anode. Electrons are emitted at the
anode and accelerate under an applied uniform electric field $E_0$ with
an acceleration fo $1ms^{-2}$. When they reach a critical velocity $u_0$
they can undergo fruitful collisions with atoms that get excited and
relax immediately to emit photons. Electrons that reach the anode are
lost. This model is simulated for $nk$ turns, each beginning with an
injection of electrons. The injection of electrons is modeled as follows
as a normally distributed random variable.

\begin{equation}
m = N + Msig \cdot X
\end{equation}

where X is a normally distributed random variable. Thus

\begin{align}
E(m) &=N + Msig \cdot E(X) = N\\
Var(m) &= Msig^2Var(X)\\
\end{align}

Also $E(X)$ is chosen to be $0$. The integer part of $m$ is chosen to be
the number of electrons injected. Msig determines the variance of the
random variable m.

\section{Code and results}\label{code-and-results}

The necessary libraries are imported and the default size for images is
set.
   \end{multicols}
    \begin{Verbatim}[commandchars=\\\{\}]
         \PY{k+kn}{from} \PY{n+nn}{\PYZus{}\PYZus{}future\PYZus{}\PYZus{}} \PY{k+kn}{import} \PY{n}{division}
         \PY{o}{\PYZpc{}} \PY{n}{matplotlib} \PY{n}{inline}
         \PY{k+kn}{import} \PY{n+nn}{matplotlib.pyplot} \PY{k+kn}{as} \PY{n+nn}{plt}
         \PY{k+kn}{from} \PY{n+nn}{matplotlib} \PY{k+kn}{import} \PY{n}{cm}\PY{p}{,} \PY{n}{colors}
         \PY{k+kn}{import} \PY{n+nn}{numpy} \PY{k+kn}{as} \PY{n+nn}{np}
         \PY{k+kn}{import} \PY{n+nn}{sys}
         \PY{n}{size}\PY{o}{=}\PY{p}{(}\PY{l+m+mi}{10}\PY{p}{,}\PY{l+m+mi}{8}\PY{p}{)}
\end{Verbatim}


    \begin{Verbatim}[commandchars=\\\{\}]
         \PY{c+c1}{\PYZsh{} defining the necessary functions}
\end{Verbatim}

\begin{multicols}{2}
    The default parameters are set as given below. Alternately the same can
be taken from commandline arguments. The default length of the tube is
taken to be $n=100$ units. Further, the probability that a sufficiently
energitic electron undergoes a collision in a turn is given by $p$.
\end{multicols}
    \begin{Verbatim}[commandchars=\\\{\}]
         \PY{c+c1}{\PYZsh{} defining the constants }
         
         \PY{c+c1}{\PYZsh{} defaults}
         
         \PY{n}{n}\PY{o}{=} \PY{l+m+mi}{100}
         \PY{n}{M}\PY{o}{=} \PY{l+m+mi}{5}
         \PY{n}{nk}\PY{o}{=} \PY{l+m+mi}{500}
         \PY{n}{u0}\PY{o}{=} \PY{l+m+mi}{7}
         \PY{n}{p}\PY{o}{=} \PY{l+m+mf}{0.5}
         \PY{n}{Msig}\PY{o}{=}\PY{l+m+mi}{2}
\end{Verbatim}


    \begin{Verbatim}[commandchars=\\\{\}]
        \PY{c+c1}{\PYZsh{} \PYZsh{} considering the command line arguments}
        \PY{c+c1}{\PYZsh{} if len(sys.argv)\PYZgt{}0:}
        \PY{c+c1}{\PYZsh{}     n= int(sys.argv[1])}
        \PY{c+c1}{\PYZsh{}     M= int(sys.argv[2])}
        \PY{c+c1}{\PYZsh{}     nk= int(sys.argv[3])}
        \PY{c+c1}{\PYZsh{}     u0= int(sys.argv[4])}
        \PY{c+c1}{\PYZsh{}     p= int(sys.argv[5])}
\end{Verbatim}


\begin{multicols}{2}


The length list, in which each element represents one electron, is made
sufficiently large. Other arrays are defined as follows\\
xx - electron position\\
u  - electron velocity\\
dx - electron displacement in one turn\\
I  -list of positions of every photon ever emitted\\
X  - list of positions of electrons that existed at the end of every turn\\
V  - the electron velocities corresponding to X\\

\end{multicols}


    \begin{Verbatim}[commandchars=\\\{\}]
         \PY{n}{length}\PY{o}{=} \PY{n}{n}\PY{o}{*}\PY{n}{M} \PY{c+c1}{\PYZsh{} shouldn\PYZsq{}t it be nk*M?}
         
         \PY{c+c1}{\PYZsh{} electron information}
         
         \PY{n}{xx}\PY{o}{=} \PY{n}{np}\PY{o}{.}\PY{n}{zeros}\PY{p}{(}\PY{p}{(}\PY{n}{length}\PY{p}{)}\PY{p}{)}   \PY{c+c1}{\PYZsh{} electron position}
         \PY{n}{u}\PY{o}{=} \PY{n}{np}\PY{o}{.}\PY{n}{zeros}\PY{p}{(}\PY{p}{(}\PY{n}{length}\PY{p}{)}\PY{p}{)}    \PY{c+c1}{\PYZsh{} electron velocity}
         \PY{n}{dx}\PY{o}{=} \PY{n}{np}\PY{o}{.}\PY{n}{zeros}\PY{p}{(}\PY{p}{(}\PY{n}{length}\PY{p}{)}\PY{p}{)}   \PY{c+c1}{\PYZsh{} disp. in current turn}
         
         \PY{c+c1}{\PYZsh{} extra info i don\PYZsq{}t really understand why i\PYZsq{}m defining}
         
         \PY{n}{I}\PY{o}{=}\PY{p}{[}\PY{p}{]}
         \PY{n}{X}\PY{o}{=}\PY{p}{[}\PY{p}{]}
         \PY{n}{V}\PY{o}{=}\PY{p}{[}\PY{p}{]}
                     
\end{Verbatim}

\begin{multicols}{2}
    The code does the following- - ii finds the positions of electrons in
the array xx which exist. $0$ in the xx array indicates the
non-existence of an electron. Each existing electron undergoes motion
according to the equations


\begin{align}
dx &= u_0t + \frac{1}{2}at^2\\
u &= u_0 + at\\
\end{align}

\begin{itemize}
\itemsep1pt\parskip0pt\parsep0pt
\item
  Due to this motion, some electrons might reach the anode and get
  absorbed. Correspondingly, their xx, u, dx values are set to zero.
  Which basically translates to their disappearance from our simulation
  universe.
\item
  Some electons obtain critical velocity $u_0$. Of these a few undergo
  collisions leading to immediate photon emission. This is modeled as a
  uniform random variable. Each energitic electron is assigned a random
  number (following a uniform distribution) and each electron collides
  if this number is lesser than $p$ which means the probability of a
  collsiion is $p$.
\item
  The velocities of these colliding electrons are set to zero as they
  lose all their energy in the collision. Since the collision might
  occur at any position betweet $x_i$ and $x_{i+1}$ (where $x_i$ is the
  position of some colliding electron corresponding to the $i^{th}$
  turn) the new position is updated by subtracting some random fraction
  of the displacement that had been added to the position of the
  colliding electron in that turn.
\item
  The positions where the collisions occur are appended to the list $I$.
\item
  New electrons are generated as explained in Section 1 and fill the
  empty positionsin the xx lists.
\item 
  As can be seen, the index ii is computed only once. It is computed once before the creation of the loop to define the variable ii. 
\end{itemize}
\end{multicols}
       \begin{Verbatim}[commandchars=\\\{\}]
         \PY{n}{ii}\PY{o}{=} \PY{n}{np}\PY{o}{.}\PY{n}{where}\PY{p}{(}\PY{n}{xx}\PY{o}{\PYZgt{}}\PY{l+m+mi}{0}\PY{p}{)}
         \PY{k}{for} \PY{n}{\PYZus{}} \PY{o+ow}{in} \PY{n+nb}{range}\PY{p}{(}\PY{l+m+mi}{0}\PY{p}{,}\PY{n}{nk}\PY{p}{)}\PY{p}{:}
         \PY{c+c1}{\PYZsh{} to find where electrons are active}
         \PY{c+c1}{\PYZsh{}    ii= np.where(xx\PYZgt{}0)   }
             \PY{n}{dx}\PY{p}{[}\PY{n}{ii}\PY{p}{]}\PY{o}{=} \PY{n}{u}\PY{p}{[}\PY{n}{ii}\PY{p}{]}\PY{o}{+}\PY{l+m+mf}{0.5}
             \PY{n}{xx}\PY{p}{[}\PY{n}{ii}\PY{p}{]}\PY{o}{=} \PY{n}{xx}\PY{p}{[}\PY{n}{ii}\PY{p}{]}\PY{o}{+}\PY{n}{dx}\PY{p}{[}\PY{n}{ii}\PY{p}{]}
             \PY{n}{u}\PY{p}{[}\PY{n}{ii}\PY{p}{]}\PY{o}{=} \PY{n}{u}\PY{p}{[}\PY{n}{ii}\PY{p}{]}\PY{o}{+}\PY{l+m+mi}{1}
             \PY{n}{anode}\PY{o}{=} \PY{n}{np}\PY{o}{.}\PY{n}{where}\PY{p}{(}\PY{n}{xx}\PY{o}{\PYZgt{}}\PY{n}{n}\PY{p}{)}
             \PY{n}{xx}\PY{p}{[}\PY{n}{anode}\PY{p}{]}\PY{o}{=} \PY{l+m+mi}{0}
             \PY{n}{dx}\PY{p}{[}\PY{n}{anode}\PY{p}{]}\PY{o}{=} \PY{l+m+mi}{0}
             \PY{n}{u}\PY{p}{[}\PY{n}{anode}\PY{p}{]}\PY{o}{=} \PY{l+m+mi}{0}
             \PY{n}{kk}\PY{o}{=} \PY{n}{np}\PY{o}{.}\PY{n}{where}\PY{p}{(}\PY{n}{u}\PY{o}{\PYZgt{}}\PY{n}{u0}\PY{p}{)} \PY{c+c1}{\PYZsh{} electrons with energy}
             \PY{n}{ll}\PY{o}{=} \PY{n}{np}\PY{o}{.}\PY{n}{where}\PY{p}{(}\PY{n}{np}\PY{o}{.}\PY{n}{random}\PY{o}{.}\PY{n}{random}\PY{p}{(}\PY{n+nb}{len}\PY{p}{(}\PY{n}{kk}\PY{p}{[}\PY{l+m+mi}{0}\PY{p}{]}\PY{p}{)}\PY{p}{)}\PY{o}{\PYZgt{}}\PY{n}{p}\PY{p}{)}
             \PY{n}{kl}\PY{o}{=} \PY{n}{np}\PY{o}{.}\PY{n}{array}\PY{p}{(}\PY{n}{kk}\PY{p}{)}\PY{p}{[}\PY{l+m+mi}{0}\PY{p}{]}\PY{p}{[}\PY{n}{ll}\PY{p}{]}  \PY{c+c1}{\PYZsh{} which electrons will ionize}
             \PY{n}{u}\PY{p}{[}\PY{n}{kl}\PY{p}{]}\PY{o}{=} \PY{l+m+mi}{0}
             \PY{n}{rho}\PY{o}{=} \PY{n}{np}\PY{o}{.}\PY{n}{random}\PY{o}{.}\PY{n}{random}\PY{p}{(}\PY{l+m+mi}{1}\PY{p}{)}\PY{p}{[}\PY{l+m+mi}{0}\PY{p}{]}
             \PY{n}{xx}\PY{p}{[}\PY{n}{kl}\PY{p}{]}\PY{o}{=} \PY{n}{xx}\PY{p}{[}\PY{n}{kl}\PY{p}{]}\PY{o}{\PYZhy{}} \PY{n}{dx}\PY{p}{[}\PY{n}{kl}\PY{p}{]}\PY{o}{*}\PY{n}{rho} \PY{c+c1}{\PYZsh{} minus cos xx has already been updated with an addition of dx in the loop}
             \PY{n}{I}\PY{o}{.}\PY{n}{extend}\PY{p}{(}\PY{n}{xx}\PY{p}{[}\PY{n}{kl}\PY{p}{]}\PY{p}{)}
             \PY{n}{m}\PY{o}{=} \PY{n+nb}{int}\PY{p}{(}\PY{n}{M} \PY{o}{+} \PY{n}{Msig}\PY{o}{*}\PY{n}{np}\PY{o}{.}\PY{n}{random}\PY{o}{.}\PY{n}{randn}\PY{p}{(}\PY{p}{)}\PY{p}{)} \PY{c+c1}{\PYZsh{} mean  5 standard deviation 2}
             \PY{n}{new}\PY{o}{=} \PY{n}{np}\PY{o}{.}\PY{n}{where}\PY{p}{(}\PY{n}{xx}\PY{o}{==}\PY{l+m+mi}{0}\PY{p}{)}
             \PY{n}{xx}\PY{p}{[}\PY{n}{new}\PY{p}{[}\PY{l+m+mi}{0}\PY{p}{]}\PY{p}{[}\PY{p}{:}\PY{n}{m}\PY{p}{]}\PY{p}{]}\PY{o}{=}\PY{l+m+mi}{1}
             \PY{n}{ii}\PY{o}{=} \PY{n}{np}\PY{o}{.}\PY{n}{where}\PY{p}{(}\PY{n}{xx}\PY{o}{\PYZgt{}}\PY{l+m+mi}{0}\PY{p}{)}
             \PY{n}{X}\PY{o}{.}\PY{n}{extend}\PY{p}{(}\PY{n}{xx}\PY{p}{[}\PY{n}{ii}\PY{p}{]}\PY{p}{)}
             \PY{n}{V}\PY{o}{.}\PY{n}{extend}\PY{p}{(}\PY{n}{u}\PY{p}{[}\PY{n}{ii}\PY{p}{]}\PY{p}{)}
           
             
             
         
             
\end{Verbatim}


    \begin{Verbatim}[commandchars=\\\{\}]
         \PY{n}{plt}\PY{o}{.}\PY{n}{figure}\PY{p}{(}\PY{l+m+mi}{0}\PY{p}{,}\PY{n}{figsize}\PY{o}{=}\PY{n}{size}\PY{p}{)}
         \PY{n}{plt}\PY{o}{.}\PY{n}{title}\PY{p}{(}\PY{l+s+s2}{\PYZdq{}}\PY{l+s+s2}{Electron density}\PY{l+s+s2}{\PYZdq{}}\PY{p}{,}\PY{n}{fontsize}\PY{o}{=}\PY{l+m+mi}{18}\PY{p}{)}
         \PY{n}{plt}\PY{o}{.}\PY{n}{xlabel}\PY{p}{(}\PY{l+s+s2}{\PYZdq{}}\PY{l+s+s2}{Position}\PY{l+s+s2}{\PYZdq{}}\PY{p}{,}\PY{n}{fontsize}\PY{o}{=}\PY{l+m+mi}{18}\PY{p}{)}
         \PY{n}{plt}\PY{o}{.}\PY{n}{ylabel}\PY{p}{(}\PY{l+s+s2}{\PYZdq{}}\PY{l+s+s2}{Number of electrons to have ever occupied the bin}\PY{l+s+s2}{\PYZdq{}}\PY{p}{,}\PY{n}{fontsize}\PY{o}{=}\PY{l+m+mi}{18}\PY{p}{)}
         \PY{n}{plt}\PY{o}{.}\PY{n}{grid}\PY{p}{(}\PY{n+nb+bp}{True}\PY{p}{)}
         \PY{n}{plt}\PY{o}{.}\PY{n}{hist}\PY{p}{(}\PY{n}{X}\PY{p}{,}\PY{n}{np}\PY{o}{.}\PY{n}{arange}\PY{p}{(}\PY{l+m+mi}{0}\PY{p}{,}\PY{l+m+mi}{101}\PY{p}{,}\PY{l+m+mi}{1}\PY{p}{)}\PY{p}{,}\PY{n}{color}\PY{o}{=}\PY{l+s+s2}{\PYZdq{}}\PY{l+s+s2}{white}\PY{l+s+s2}{\PYZdq{}}\PY{p}{)}
         \PY{n}{plt}\PY{o}{.}\PY{n}{show}\PY{p}{(}\PY{p}{)}    
         \PY{n}{plt}\PY{o}{.}\PY{n}{close}\PY{p}{(}\PY{p}{)}
\end{Verbatim}


    \begin{center}
    \adjustimage{max size={0.9\linewidth}{0.9\paperheight}}{output_10_0.png}
    \end{center}
    { \hspace*{\fill} \\}
\begin{multicols}{2}
    The following histogram shows I vs position. This is a representation of
the average value of the number of photons emitted between any two
consecutive integral positions along hte tubelight and thus a
representation of the average intensity.
\end{multicols}
    \begin{Verbatim}[commandchars=\\\{\}]
         \PY{n}{plt}\PY{o}{.}\PY{n}{figure}\PY{p}{(}\PY{l+m+mi}{1}\PY{p}{,}\PY{n}{figsize}\PY{o}{=}\PY{n}{size}\PY{p}{)}
         \PY{n}{plt}\PY{o}{.}\PY{n}{title}\PY{p}{(}\PY{l+s+s2}{\PYZdq{}}\PY{l+s+s2}{Intensity (average) of the tubelight vs position}\PY{l+s+s2}{\PYZdq{}}\PY{p}{,} \PY{n}{fontsize}\PY{o}{=}\PY{l+m+mi}{18}\PY{p}{)}
         \PY{n}{plt}\PY{o}{.}\PY{n}{xlabel}\PY{p}{(}\PY{l+s+s2}{\PYZdq{}}\PY{l+s+s2}{Position}\PY{l+s+s2}{\PYZdq{}}\PY{p}{,} \PY{n}{fontsize}\PY{o}{=}\PY{l+m+mi}{18}\PY{p}{)}
         \PY{n}{plt}\PY{o}{.}\PY{n}{ylabel}\PY{p}{(}\PY{l+s+s2}{\PYZdq{}}\PY{l+s+s2}{Intensity/ Number of photons emitted}\PY{l+s+s2}{\PYZdq{}}\PY{p}{,}\PY{n}{fontsize}\PY{o}{=}\PY{l+m+mi}{18}\PY{p}{)}
         \PY{n}{plt}\PY{o}{.}\PY{n}{grid}\PY{p}{(}\PY{n+nb+bp}{True}\PY{p}{)}
         \PY{n}{ret}\PY{o}{=}\PY{n}{plt}\PY{o}{.}\PY{n}{hist}\PY{p}{(}\PY{n}{I}\PY{p}{,}\PY{n}{np}\PY{o}{.}\PY{n}{arange}\PY{p}{(}\PY{l+m+mi}{0}\PY{p}{,}\PY{l+m+mi}{101}\PY{p}{,}\PY{l+m+mi}{1}\PY{p}{)}\PY{p}{,}\PY{n}{color}\PY{o}{=}\PY{l+s+s2}{\PYZdq{}}\PY{l+s+s2}{white}\PY{l+s+s2}{\PYZdq{}}\PY{p}{)}
         \PY{n}{vals}\PY{o}{=}\PY{n}{ret}\PY{p}{[}\PY{l+m+mi}{0}\PY{p}{]}
         \PY{n}{vals}\PY{o}{=}\PY{l+m+mi}{1}\PY{o}{\PYZhy{}}\PY{p}{(}\PY{n}{vals}\PY{o}{/}\PY{n+nb}{max}\PY{p}{(}\PY{n}{vals}\PY{p}{)}\PY{p}{)}
         \PY{n}{norm} \PY{o}{=} \PY{n}{colors}\PY{o}{.}\PY{n}{Normalize}\PY{p}{(}\PY{n}{vals}\PY{o}{.}\PY{n}{min}\PY{p}{(}\PY{p}{)}\PY{p}{,} \PY{n}{vals}\PY{o}{.}\PY{n}{max}\PY{p}{(}\PY{p}{)}\PY{p}{)}
         \PY{k}{for} \PY{n}{thisfrac}\PY{p}{,} \PY{n}{thispatch} \PY{o+ow}{in} \PY{n+nb}{zip}\PY{p}{(}\PY{n}{vals}\PY{p}{,} \PY{n}{ret}\PY{p}{[}\PY{l+m+mi}{2}\PY{p}{]}\PY{p}{)}\PY{p}{:}
             \PY{n}{color} \PY{o}{=} \PY{n}{cm}\PY{o}{.}\PY{n}{Greys}\PY{p}{(}\PY{n}{norm}\PY{p}{(}\PY{n}{thisfrac}\PY{p}{)}\PY{p}{)}
             \PY{n}{thispatch}\PY{o}{.}\PY{n}{set\PYZus{}facecolor}\PY{p}{(}\PY{n}{color}\PY{p}{)}
         \PY{n}{plt}\PY{o}{.}\PY{n}{show}\PY{p}{(}\PY{p}{)}    
         \PY{n}{plt}\PY{o}{.}\PY{n}{close}\PY{p}{(}\PY{p}{)}  
\end{Verbatim}


    \begin{center}
    \adjustimage{max size={0.9\linewidth}{0.9\paperheight}}{output_12_0.png}
    \end{center}
    { \hspace*{\fill} \\}
\begin{multicols}{2}
    We plot the phase space of the electrons and also make a table of the
centers of the bins and the intensity (number of photons) values.
\end{multicols}
    \begin{Verbatim}[commandchars=\\\{\}]
         \PY{n}{plt}\PY{o}{.}\PY{n}{figure}\PY{p}{(}\PY{l+m+mi}{2}\PY{p}{,}\PY{n}{figsize}\PY{o}{=}\PY{n}{size}\PY{p}{)}
         \PY{n}{plt}\PY{o}{.}\PY{n}{xlabel}\PY{p}{(}\PY{l+s+s2}{\PYZdq{}}\PY{l+s+s2}{X}\PY{l+s+s2}{\PYZdq{}}\PY{p}{,}\PY{n}{fontsize}\PY{o}{=}\PY{l+m+mi}{18}\PY{p}{)}
         \PY{n}{plt}\PY{o}{.}\PY{n}{title}\PY{p}{(}\PY{l+s+s2}{\PYZdq{}}\PY{l+s+s2}{Electron phase space}\PY{l+s+s2}{\PYZdq{}}\PY{p}{,}\PY{n}{fontsize}\PY{o}{=}\PY{l+m+mi}{18}\PY{p}{)}
         \PY{n}{plt}\PY{o}{.}\PY{n}{ylabel}\PY{p}{(}\PY{l+s+s2}{\PYZdq{}}\PY{l+s+s2}{V}\PY{l+s+s2}{\PYZdq{}}\PY{p}{,}\PY{n}{fontsize}\PY{o}{=}\PY{l+m+mi}{18}\PY{p}{)}
         \PY{n}{plt}\PY{o}{.}\PY{n}{grid}\PY{p}{(}\PY{n+nb+bp}{True}\PY{p}{)}
         \PY{n}{plt}\PY{o}{.}\PY{n}{plot}\PY{p}{(}\PY{n}{X}\PY{p}{,}\PY{n}{V}\PY{p}{,}\PY{l+s+s2}{\PYZdq{}}\PY{l+s+s2}{bx}\PY{l+s+s2}{\PYZdq{}}\PY{p}{)}
         \PY{n}{plt}\PY{o}{.}\PY{n}{show}\PY{p}{(}\PY{p}{)} 
         \PY{n}{plt}\PY{o}{.}\PY{n}{close}\PY{p}{(}\PY{p}{)}
\end{Verbatim}


    \begin{center}
    \adjustimage{max size={0.9\linewidth}{0.9\paperheight}}{output_14_0.png}
    \end{center}
    { \hspace*{\fill} \\}
\begin{multicols}{2}
    The following is a 2D extension of the one dimensional model wherein the
intensity value along $y$ axis has been kept the same at a given $x$.
\end{multicols}
    \begin{Verbatim}[commandchars=\\\{\}]
         \PY{n}{intensity}\PY{o}{=}\PY{n}{np}\PY{o}{.}\PY{n}{array}\PY{p}{(}\PY{p}{[}\PY{n}{ret}\PY{p}{[}\PY{l+m+mi}{0}\PY{p}{]}\PY{p}{]}\PY{o}{*}\PY{l+m+mi}{100}\PY{p}{)}
         \PY{n}{y}\PY{p}{,}\PY{n}{x}\PY{o}{=}\PY{n}{np}\PY{o}{.}\PY{n}{meshgrid}\PY{p}{(}\PY{n}{ret}\PY{p}{[}\PY{l+m+mi}{1}\PY{p}{]}\PY{p}{[}\PY{p}{:}\PY{o}{\PYZhy{}}\PY{l+m+mi}{1}\PY{p}{]}\PY{p}{,}\PY{n}{ret}\PY{p}{[}\PY{l+m+mi}{1}\PY{p}{]}\PY{p}{[}\PY{p}{:}\PY{o}{\PYZhy{}}\PY{l+m+mi}{1}\PY{p}{]}\PY{p}{)}
         \PY{n}{plt}\PY{o}{.}\PY{n}{figure}\PY{p}{(}\PY{n}{figsize}\PY{o}{=}\PY{p}{(}\PY{l+m+mi}{14}\PY{p}{,}\PY{l+m+mi}{4}\PY{p}{)}\PY{p}{)}
         \PY{n}{plt}\PY{o}{.}\PY{n}{contourf}\PY{p}{(}\PY{n}{y}\PY{p}{,}\PY{n}{x}\PY{p}{,}\PY{o}{\PYZhy{}}\PY{n}{intensity}\PY{p}{,}\PY{n}{cmap}\PY{o}{=}\PY{n}{cm}\PY{o}{.}\PY{n}{Blues}\PY{p}{)}
         \PY{n}{plt}\PY{o}{.}\PY{n}{title}\PY{p}{(}\PY{l+s+s2}{\PYZdq{}}\PY{l+s+s2}{Tubelight, extension from 1D to 2D}\PY{l+s+s2}{\PYZdq{}}\PY{p}{,} \PY{n}{fontsize}\PY{o}{=}\PY{l+m+mi}{18}\PY{p}{)}
         \PY{n}{plt}\PY{o}{.}\PY{n}{xlabel}\PY{p}{(}\PY{l+s+s2}{\PYZdq{}}\PY{l+s+s2}{x}\PY{l+s+s2}{\PYZdq{}}\PY{p}{,} \PY{n}{fontsize}\PY{o}{=}\PY{l+m+mi}{18}\PY{p}{)}
         \PY{n}{plt}\PY{o}{.}\PY{n}{show}\PY{p}{(}\PY{p}{)}
         \PY{n}{plt}\PY{o}{.}\PY{n}{close}\PY{p}{(}\PY{p}{)}
\end{Verbatim}


    \begin{center}
    \adjustimage{max size={0.9\linewidth}{0.9\paperheight}}{output_16_0.png}
    \end{center}
    { \hspace*{\fill} \\}
\begin{multicols}{2}
The following code improves upon the previous code in the sense that it
takes into account the fact that electrnos accelerate post collision in
a given turn. Further it takes into account the fact that it is the time
that is uniformly distributed not the position of collision. The outputs
corresponding to the folowing code are shown. It is evident that a whole
new set of electron states are made accessible as th electrons begin to
show low velocities at various high values of x.
\end{multicols}

    \begin{Verbatim}[commandchars=\\\{\}]
         \PY{o}{\PYZpc{}\PYZpc{}}\PY{k}{time}
          ii= np.where(xx\PYZgt{}0) 
          for \PYZus{} in range(0,nk):
          \PYZsh{} to find where electrons are active
             
              dx[ii]= u[ii]+0.5
              xx[ii]= xx[ii]+dx[ii]
              u[ii]= u[ii]+1
              anode= np.where(xx\PYZgt{}n)
              xx[anode]= 0
              dx[anode]= 0
              u[anode]= 0
              kk= np.where(u\PYZgt{}u0) \PYZsh{} electrons with energy
              ll= np.where(np.random.random(len(kk[0]))\PYZgt{}p)
              kl= np.array(kk)[0][ll]  \PYZsh{} which electrons will ionize
              \PYZsh{}u[kl]= 0
              rho= np.random.random(1)[0]
              temp=xx[kl]
              xx[kl]= xx[kl]\PYZhy{} dx[kl]*rho\PYZhy{}0.204+0.241*rho \PYZsh{} 
              #minus cos xx has already been updated with an addition of dx in the loop
              t=np.sqrt((u[kl]\PYZhy{}1)**2 + 2*(xx[kl]\PYZhy{}(temp\PYZhy{}dx[kl])))\PYZhy{}u[kl]+1
              t=1\PYZhy{}t
              u[kl]=t
              I.extend(xx[kl])
              xx[kl]=xx[kl]+0.5*(t**2)
              m= int(M + Msig*np.random.randn()) \PYZsh{} mean  5 standard deviation 2
              new= np.where(xx==0)
              xx[new[0][:m]]=1
              ii= np.where(xx\PYZgt{}0)
              X.extend(xx[ii])
              V.extend(u[ii])
            
              
              
          
              
\end{Verbatim}

The following are the corresponding outputs.
   \begin{center}
    \adjustimage{max size={0.9\linewidth}{0.9\paperheight}}{output_11_0.png}
    \end{center}

        \begin{center}
    \adjustimage{max size={0.9\linewidth}{0.9\paperheight}}{output_13_0.png}
    \end{center}

        \begin{center}
    \adjustimage{max size={0.9\linewidth}{0.9\paperheight}}{output_15_0.png}
    \end{center}

        \begin{center}
    \adjustimage{max size={0.9\linewidth}{0.9\paperheight}}{output_17_0.png}
    \end{center}

    
    \begin{Verbatim}[commandchars=\\\{\}]
        \PY{n}{bins}\PY{o}{=}\PY{n}{ret}\PY{p}{[}\PY{l+m+mi}{1}\PY{p}{]}
        \PY{n}{bins}\PY{o}{=} \PY{l+m+mf}{0.5}\PY{o}{*}\PY{p}{(}\PY{n}{bins}\PY{p}{[}\PY{l+m+mi}{0}\PY{p}{:}\PY{o}{\PYZhy{}}\PY{l+m+mi}{1}\PY{p}{]}\PY{o}{+}\PY{n}{bins}\PY{p}{[}\PY{l+m+mi}{1}\PY{p}{:}\PY{p}{]}\PY{p}{)}
        \PY{k}{print} \PY{p}{(}\PY{l+s+s2}{\PYZdq{}}\PY{l+s+s2}{Intensity data }\PY{l+s+se}{\PYZbs{}n}\PY{l+s+s2}{\PYZdq{}}\PY{p}{)}
        \PY{k}{print} \PY{p}{(}\PY{l+s+s2}{\PYZdq{}}\PY{l+s+s2}{xpos     count}\PY{l+s+s2}{\PYZdq{}}\PY{p}{)}
        \PY{k}{for} \PY{n}{i} \PY{o+ow}{in} \PY{n+nb}{range}\PY{p}{(}\PY{n+nb}{len}\PY{p}{(}\PY{n}{bins}\PY{p}{)}\PY{p}{)}\PY{p}{:}
            \PY{k}{if} \PY{n}{bins}\PY{p}{[}\PY{n}{i}\PY{p}{]}\PY{o}{\PYZlt{}}\PY{l+m+mi}{10}\PY{p}{:}
                \PY{k}{print} \PY{n+nb}{str}\PY{p}{(}\PY{n}{bins}\PY{p}{[}\PY{n}{i}\PY{p}{]}\PY{p}{)}\PY{o}{+}\PY{l+s+s2}{\PYZdq{}}\PY{l+s+s2}{       }\PY{l+s+s2}{\PYZdq{}}\PY{o}{+}\PY{n+nb}{str}\PY{p}{(}\PY{n}{ret}\PY{p}{[}\PY{l+m+mi}{0}\PY{p}{]}\PY{p}{[}\PY{n}{i}\PY{p}{]}\PY{p}{)}
            \PY{k}{else}\PY{p}{:}
                \PY{k}{print} \PY{n+nb}{str}\PY{p}{(}\PY{n}{bins}\PY{p}{[}\PY{n}{i}\PY{p}{]}\PY{p}{)}\PY{o}{+}\PY{l+s+s2}{\PYZdq{}}\PY{l+s+s2}{      }\PY{l+s+s2}{\PYZdq{}}\PY{o}{+}\PY{n+nb}{str}\PY{p}{(}\PY{n}{ret}\PY{p}{[}\PY{l+m+mi}{0}\PY{p}{]}\PY{p}{[}\PY{n}{i}\PY{p}{]}\PY{p}{)}
\end{Verbatim}


    \begin{Verbatim}[commandchars=\\\{\}]
Intensity data 

xpos     count
0.5       0.0
1.5       0.0
2.5       0.0
3.5       0.0
4.5       0.0
5.5       0.0
6.5       0.0
7.5       0.0
8.5       0.0
9.5       0.0
10.5      0.0
11.5      0.0
12.5      0.0
13.5      0.0
14.5      0.0
15.5      0.0
16.5      0.0
17.5      0.0
18.5      0.0
19.5      0.0
20.5      0.0
21.5      0.0
22.5      0.0
23.5      0.0
24.5      0.0
25.5      81.0
26.5      212.0
27.5      328.0
28.5      152.0
29.5      119.0
30.5      123.0
31.5      170.0
32.5      139.0
33.5      86.0
34.5      201.0
35.5      54.0
36.5      68.0
37.5      65.0
38.5      50.0
39.5      69.0
40.5      71.0
41.5      55.0
42.5      31.0
43.5      30.0
44.5      21.0
45.5      84.0
46.5      30.0
47.5      28.0
48.5      31.0
49.5      34.0
50.5      28.0
51.5      55.0
52.5      48.0
53.5      145.0
54.5      65.0
55.5      63.0
56.5      103.0
57.5      79.0
58.5      72.0
59.5      76.0
60.5      100.0
61.5      87.0
62.5      178.0
63.5      53.0
64.5      83.0
65.5      67.0
66.5      73.0
67.5      68.0
68.5      85.0
69.5      63.0
70.5      87.0
71.5      67.0
72.5      95.0
73.5      78.0
74.5      30.0
75.5      48.0
76.5      59.0
77.5      89.0
78.5      41.0
79.5      70.0
80.5      64.0
81.5      80.0
82.5      72.0
83.5      59.0
84.5      55.0
85.5      57.0
86.5      51.0
87.5      83.0
88.5      95.0
89.5      92.0
90.5      77.0
91.5      63.0
92.5      57.0
93.5      47.0
94.5      51.0
95.5      41.0
96.5      29.0
97.5      32.0
98.5      16.0
99.5      8.0

    \end{Verbatim}
    
\begin{multicols}{2}
   The following are the outputs for $u_0= 5$ and $p=0.25$
\end{multicols}
    \begin{center}
    \adjustimage{max size={0.9\linewidth}{0.9\paperheight}}{X2.png}
    \end{center}

        \begin{center}
    \adjustimage{max size={0.9\linewidth}{0.9\paperheight}}{I2.png}
    \end{center}

        \begin{center}
    \adjustimage{max size={0.9\linewidth}{0.9\paperheight}}{ps2.png}
    \end{center}

        \begin{center}
    \adjustimage{max size={0.9\linewidth}{0.9\paperheight}}{tb2.png}
    \end{center}
\begin{multicols}{2}
    \section{Discussion and Conclusion-}\label{discussion-and-conclusion-}

\paragraph{}The population plot of Figure 1 indicates the number of electrons that
have been at a given position during the simulation. Since all the
electrons start at the same point the first bin has the highest peak.
This is followed by $x\approx25$ which is where most electrons have the
energy necessary for ionisation.

\paragraph{} However the plot of Figure 2 is of far greater interest. It is
effectively a plot of the number of electron atom collisions or the
number of photons emitted at a given point or in a given bin. Since most
of the electrons have insufficent energy for the first 20 - 25 units or
so, there are no bars in this region. However most of the electrons gain
sufficient energy by 25 (for the given defaults), and thus cause photon
emission. Thus there is a huge peak at this point. Electrons that don't
suffer collisions move on and perhaps do so at later stages contributing
to the further bars. The next set of peaks is noticed at around 60,
which incidentally happens to be nearly twice the initial value of
position of the first peak. This could probably be due to the face that
a majority of the electrons that suffered collisions at the position of
the first peak have gained enought energy for a second set of
collisions.

\paragraph{} The phase space further adds insight. The phase space is the set of all
possible states of a system. In this case, it is evident that no high
velocity, low position states are present. This arises from the fact
that for all low valus of position, until the first peak in intensity,
the velocity of the electrons is only increasing and there are no
collisions. Once the electrons have sufficient energy, they begin to
suffer collisions which means they end up with zero velocity at multiple
values of X and similarly at higher values of velocity for other
positions. This means that a multitude of combinations of $X_i$ and
$V_i$ are possible which is made obvious by the phase space plot.

\paragraph{} To conclude, the various scientific python libraries have been made use
of effectively to model what is a fairly complex situation which
inherently involves some randomness. Further the intensity distribution
of the tube is studied and found to match expectations.


    % Add a bibliography block to the postdoc
    \end{multicols}
    
    
    \end{document}
